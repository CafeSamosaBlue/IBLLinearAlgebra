\documentclass{article}
\usepackage{amsmath}
\usepackage{amsfonts}
\usepackage{amssymb}
\usepackage{graphicx}
\usepackage{enumitem}
\usepackage{amsfonts}
\usepackage{amssymb}
\usepackage{xparse}
\usepackage{ifthen}


\oddsidemargin=0in
\evensidemargin=0in
\textwidth=6in
\topmargin=-1in
\textheight=10in
\parskip=.07in

\parindent=0in
%\pagestyle{empty}

\newcounter{EnumPrefix}
\newcounter{EnumSuffix}
%\newenvironment{Enum}[0]{
\DeclareDocumentEnvironment{Enum}{o}{
	\ifthenelse{ \equal{#1}{resume} }{
		\begin{enumerate}[label={\footnotesize\arabic{EnumPrefix}}.\arabic*]
		\setcounter{enumi}{\value{EnumSuffix}}
	}{
		\stepcounter{EnumPrefix}\begin{enumerate}[label={\footnotesize\arabic{EnumPrefix}}.\arabic*]
	}
}{
	\setcounter{EnumSuffix}{\value{enumi}}
	\end{enumerate}
}
%\renewcommand{\labelenumi}{\arabic{EnumPrefix}.\arabic*}

\newcommand{\ul}{$\underline{\phantom{xxx}}$}
\newcommand{\ull}{\underline{\phantom{xxx}}}
\newcommand{\xh}{\hat{\bf x}}
\newcommand{\yh}{\hat{\bf y}}
\newcommand{\zh}{\hat{\bf z}}
\newcommand{\R}{\mathbb{R}}
\newcommand{\C}{\mathbb{C}}
\newcommand{\Z}{\mathbb{Z}}
\newcommand{\N}{\mathbb{N}}
\newcommand{\proj}{\mathrm{proj}}
\renewcommand{\span}{\mathrm{span}\,}
\newcommand{\rref}{\mathrm{rref}}
\newcommand{\rank}{\mathrm{rank}}
\newcommand{\nnul}{\mathrm{nullity}}
\newcommand{\mat}[1]{\begin{bmatrix}#1\end{bmatrix}}

\newtheorem{definition}{Definition}
\newtheorem{theorem}{Theorem}

\begin{document}
\section*{Complex Numbers}
	A large part of mathematics is the process of abstraction.
	For instance, numbers are an abstraction of the concept of length.
	If we had two numbers $a,b$ representing length, their product would be 
	the area of the rectangle whose sides were $a,b$.  Strictly speaking, with
	this interpretation, $ab$ is an area and $a,b$ are lengths and they cannot be
	compared (this is what the Greeks thought), but we have come to know numbers as
	a more abstract concept.  After all, we can write both $a,b$ and $ab$ using decimals,
	so why shouldn't we compare them?

	Taking this idea to solving polynomials gives us complex numbers.  The solutions
	to $x^2=a$ are $\pm\sqrt{a}$ formally, but if I said $a=-4$, we might claim that $\sqrt{-4}$
	doesn't make any sense.  However, with a simple definition, we can make
	a larger class of numbers where $\sqrt{-4}$ fits in perfectly.

	\begin{definition}
		The imaginary unit $i$ is defined as a solution to $x^2=-1$.  That
		is $i^2=-1$.
	\end{definition}
	\begin{definition}
		A \emph{complex number} is a number $a+bi$ where $a$ is the \emph{real part}
		and $b$ is the \emph{imaginary part}.
	\end{definition}

	You can think of complex numbers as vectors with two components, a real one
	and an imaginary one.  You add complex numbers as you would vectors.

	Compute the following:
	\begin{Enum}
		\item $(1+2i)$+$(2-7i)$
		\item $i-(6+i)$
		\item $24(3+i)-\frac{1}{2}(2+2i)$
	\end{Enum}

	Unlike vectors however, complex numbers can be multiplied.  Multiplication
	follows all the same rules as for real numbers, except $(i)(i)=-1$.  So,
	for example, $i(2-3i)=2i-3i^2=3+2i$.

	Compute the following products:
	\begin{Enum}
		\item $i(7i)$
		\item $(3+i)(2+4i)$
		\item $(7+i)(7-i)$
	\end{Enum}

	\begin{definition}
		The \emph{conjugate} of a complex number $a+bi$ is written with a bar over
		top the number and is
		\[
			\overline{a+bi}=a-bi.
		\]
	\end{definition}

	If $c$ is a complex number then $c\bar c$ is always a real number.  This
	means that while $1/c$ has complex numbers in the denominator
	of a fraction, $1/c=\bar c/(c\bar c)$ does not.  Now you know how to divide
	by a complex number!

	Compute the following:
	\begin{Enum}
		\item $\overline{2-7i}$
		\item $\bar i$
		\item $1/(2-7i)$
		\item $(4+3i)/(-5+i)$
	\end{Enum}

	Recall that for the square-root function $\sqrt{ab}=\sqrt{a}\sqrt{b}$.
	Thus for example $\sqrt{-4}=\sqrt{4}\sqrt{-1}=2i$.  Use this and your
	knowledge of the quadratic formula to solve the following:
	\begin{Enum}
		\item $x^2+3x+9=0$
		\item $-x^2-2=1$
		\item $-x^2+5x-12=7$
	\end{Enum}

	The amazing thing about complex numbers is that just by introducing the
	solution to the equation $x^2=-1$ we can now write the solutions
	to any $n$-degree polynomial as complex numbers!
	\begin{theorem}\label{t}
		If $p(x)=a_nx^n+a_{n-1}x^{n-1}+\cdots+a_1x+a_0$ then we can factor $p$
		as
		\[
			p(x)=(x-c_1)(x-c_2)\cdots (x-c_n)
		\]
		where $c_1,\ldots,c_n$ are complex numbers.
	\end{theorem}

	Theorem \ref{t} will be important as we start solving equations
	involving determinants, but you can just think of Theorem \ref{t} as
	telling us that we always have $n$ solutions to an $n$-degree polynomial
	(though the solutions may not be all distinct).

\subsection*{Complex Numbers and Matrices}
	Using complex numbers as entries in a matrix allows us to represent things
	we couldn't before.

	Compute the following:
	\begin{Enum}
		\item $\mat{1 &-i}\mat{1\\i}$
		\item $\mat{-i&i\\1&1}\mat{i&1\\-i&1}$
		\item $\mat{-i&i\\1&1}\mat{-i&0\\0&i}\mat{i&1\\-i&1}$
		\item Compute $i\mat{i\\1}$ and compare it
			with $\mat{0&-1\\1&0}\mat{i\\1}$.  What do you notice? (How
			does this relate to eigen values and eigen vectors?)
	\end{Enum}


\end{document}
