\documentclass[red]{tutorial}
\usepackage[no-math]{fontspec}
\usepackage{xpatch}
	\renewcommand{\ttdefault}{ul9}
	\xpatchcmd{\ttfamily}{\selectfont}{\fontencoding{T1}\selectfont}{}{}
	\DeclareTextCommand{\nobreakspace}{T1}{\leavevmode\nobreak\ }
\usepackage{polyglossia} % English please
	\setdefaultlanguage[variant=us]{english}
%\usepackage[charter,cal=cmcal]{mathdesign} %different font
%\usepackage{avant}
\usepackage{microtype} % Less badboxes


\usepackage[charter,cal=cmcal]{mathdesign} %different font
%\usepackage{euler}
 
\usepackage{blindtext}
\usepackage{calc, ifthen, xparse, xspace}
\usepackage{makeidx}
\usepackage[hidelinks, urlcolor=blue]{hyperref}   % Internal hyperlinks
\usepackage{mathtools} % replaces amsmath
\usepackage{bbm} %lower case blackboard font
\usepackage{amsthm, bm}
\usepackage{thmtools} % be able to repeat a theorem
\usepackage{thm-restate}
\usepackage{graphicx}
\usepackage{xcolor}
\usepackage{multicol}
\usepackage{fnpct} % fancy footnote spacing

 
\newcommand{\xh}{{{\mathbf e}_1}}
\newcommand{\yh}{{{\mathbf e}_2}}
\newcommand{\zh}{{{\mathbf e}_3}}
\newcommand{\R}{\mathbb{R}}
\newcommand{\Z}{\mathbb{Z}}
\newcommand{\N}{\mathbb{N}}
\newcommand{\proj}{\mathrm{proj}}
\newcommand{\Proj}{\mathrm{proj}}
\newcommand{\Perp}{\mathrm{perp}}
\renewcommand{\span}{\mathrm{span}\,}
\newcommand{\Span}{\mathrm{span}\,}
\newcommand{\Img}{\mathrm{img}\,}
\newcommand{\Null}{\mathrm{null}\,}
\newcommand{\Range}{\mathrm{range}\,}
\newcommand{\rref}{\mathrm{rref}}
\newcommand{\rank}{\mathrm{rank}}
\newcommand{\Rank}{\mathrm{rank}}
\newcommand{\nnul}{\mathrm{nullity}}
\newcommand{\mat}[1]{\begin{bmatrix}#1\end{bmatrix}}
\newcommand{\chr}{\mathrm{char}}
\renewcommand{\d}{\mathrm{d}}


\theoremstyle{definition}
\newtheorem{example}{Example}[section]
\newtheorem{defn}{Definition}[section]

\theoremstyle{theorem}
\newtheorem{thm}{Theorem}[section]

\pgfkeys{/tutorial,
	name={Tutorial 5},
	author={Jason Siefken},
	course={MAT 223},
	date={February~23--March~2},
	term={Spring 2019},
	title={Change of Basis}
	}

\begin{document}
	\begin{tutorial}

		\begin{objectives}
	In this tutorial you will examine the pros and cons of different bases.

	These problems relate to the following course learning objectives:
	\textit{Write vectors in
different bases and pick an appropriate basis when working on problems}.
		\end{objectives}


\subsection*{Problems}

\begin{enumerate}
	\item Write down the definition of what it means for a set $\mathcal B$ to be a \emph{basis}
		for the subspace $V$.
	\item Let $\mathcal E=\{\vec e_1,\vec e_2,\vec e_3\}$, $\mathcal B=\left\{\mat{1\\-1\\0},\mat{0\\1\\0},\mat{2\\-2\\1}\right\}$,
		and $\mathcal C=\left\{\mat{1\\-1\\1},\mat{0\\1\\1},\mat{2\\-2\\3}\right\}$
		be ordered bases, and let $\vec v=4\vec e_1-4\vec e_2+2\vec e_3$. 
	\begin{enumerate}
		\item Are the vectors in $\mathcal B$ and $\mathcal C$ written in a basis? If so, which one(s)?
			What about the vectors in $\mathcal E$?\
		\item Compute $[\vec v]_{\mathcal E}$, $[\vec v]_{\mathcal B}$, and $[\vec v]_{\mathcal C}$.
		\item Compute $[7\vec v]_{\mathcal E}$, $[7\vec v]_{\mathcal B}$, and $[7\vec v]_{\mathcal C}$. \emph{Hint: look
			at what you've already done; you might not have to do much more work.}
		\item Suppose, during a scientific experiment, you repeatedly measure multiples of the vector $\vec v$. Which
			basis would you prefer to write down your measurements in? Why?
	\end{enumerate}
\item Let $\mathcal E=\left\{\vec e_1, \vec e_2\right\}$ and $\mathcal B=\left\{\mat{1\\0},\mat{1\\1}\right\}$
		be bases for $\R^2$ and let $R:\R^2\to\R^2$ be rotation counter-clockwise by 90$^\circ$.
		\begin{enumerate}
			\item Find a matrix, $P$, for $R$ in the basis $\mathcal E$. That is $P[\vec x]_{\mathcal E} = [R\vec x]_{\mathcal E}$.
			\item Find a matrix, $Q$, for $R$ in the basis $\mathcal B$. That is $Q[\vec x]_{\mathcal B} = [R\vec x]_{\mathcal B}$.
		\end{enumerate}
	\item In math, a real number can have infinitely many digits. In a computer, however, there is limited space, so a computer
		will \emph{truncate} numbers it stores\footnote{ \emph{Truncate} means erasing digits after a certain point. So $3.14159\ldots$ might become
		$3.141$.}. Consider the bases
		\[
			\mathcal E_1=\left\{\mat{1\\0},\mat{1\\1}\right\}\qquad
			\mathcal E_2=\left\{\mat{1\\0},\mat{1\\.1}\right\}\qquad
			\mathcal E_3=\left\{\mat{1\\0},\mat{1\\.01}\right\}\qquad
			\mathcal E_4=\left\{\mat{1\\0},\mat{1\\.001}\right\}\qquad\cdots
		\]
		and the vector $\vec v=\mat{1\\0.1234567\ldots}$, written in the standard basis. Suppose your computer can only store two decimal places
		(after the decimal point) when it stores a number. Which basis should you use to represent $\vec v$? Why? 
	
\end{enumerate}
	\end{tutorial}


	\begin{solutions}
		\begin{enumerate}
			\item $\mathcal B$ is a basis for $V$ if it is a linearly independent set of
				vectors that spans $V$.
			\item \begin{enumerate}
					\item Yes. The vectors in $\mathcal B$ and $\mathcal C$ are written in
						the standard basis. The vectors in $\mathcal E$ are not written in
						any basis---they are referred to directly by name.
					\item $[\vec v]_{\mathcal E} = \mat{4\\-4\\2}$,
						$[\vec v]_{\mathcal B} = \mat{0\\0\\2}$, and
						$[\vec v]_{\mathcal C} = \mat{8\\0\\-2}$.
					\item $[7\vec v]_{\mathcal E} = \mat{28\\-28\\14}$,
						$[7\vec v]_{\mathcal B} = \mat{0\\0\\14}$, and
						$[7\vec v]_{\mathcal C} = \mat{56\\0\\-14}$.
					\item I would pick basis $\mathcal B$ since every multiple of $\vec v$ written
						in the basis $\mathcal B$ takes the form $\mat{0\\0\\t}$, so there are
						fewer numbers to write down.

			\end{enumerate}
			\item \begin{enumerate}
					\item $[R]_{\mathcal E} = P=\mat{0&-1\\1&0}$.
					\item $[R]_{\mathcal B} = Q=\mat{-1&-2\\1&1}$.
			\end{enumerate}

			\item Computing, we see
				\[
					[\vec v]_{\mathcal E_1} = \mat{0.8765\ldots\\0.1234\ldots}\quad
					[\vec v]_{\mathcal E_2} = \mat{-0.2345\ldots\\1.2345\ldots}\quad
					[\vec v]_{\mathcal E_3} = \mat{-11.3456\ldots\\12.3456\ldots}\quad
					[\vec v]_{\mathcal E_4} = \mat{-122.4567\ldots\\123.4567\ldots}
				\]
			Given the pattern, it might be best to pick $\mathcal E_i$ for $i$ as large as possible.
				However, if the computer were to also store the basis in standard form, 
				$\mathcal E_i$ for $i\geq 4$ couldn't be stored. Thus, the most accuracy we
				could get is by using $\mathcal E_3$.
		\end{enumerate}
	
	\end{solutions}
	\begin{instructions}

\subsection*{Learning Objectives}
	Students need to be able to\ldots
	\begin{itemize}
		\item Write vectors in multiple bases
		\item Write matrices for linear transformations in multiple bases
		\item Explain the pros and cons for one basis over another in a particular problem
	\end{itemize}

\subsection*{Context}
	In class, students have gone over bases, subspaces, dimension, linear transformations, and
		matrix transformations. In this class, we only work with $\R^n$ and subspaces of $\R^n$,
		so we won't be talking about bases for abstract spaces. 
	
	Bases are especially hard for students---they're used to thinking of lists of numbers
		as the actual vectors instead of representations of the vectors. They need
		a lot of hand-holding to make this transition.

\subsection*{What to Do}
	Start by explaining to students that bases and linear transformations are \emph{the}
		two big ideas from linear algebra and that today we're going to focus on bases.
		Further tell them that up till now we've been considering vectors and lists
		of numbers as interchangeable, but now we think of lists of numbers
		as a \emph{representation} of a vector instead of a true vector (kind of like
		the symbol ``4'' is not literally four, it is a graphical representation of
		the abstract idea of the number four).
	
	Proceed as usual, asking students to form small groups and start working.
		Again, this tutorial starts with a definition question, which they all need to actually write out.
		The meat of this tutorial is problem \#2.
		When most groups are on \#2(c), have a class discussion on 2(a) and 2(b), to
		make sure everyone is on the same page.


	Remember, the point of this tutorial (and all tutorials) is not to get through all the problems.
	It is to get practice with difficult and new mathematics. Don't cut thinking time short trying
	to get through more problems!

	8 minutes before the end of class, pick a problem that most groups have started, or a problem
		that they've finished but you think needs more attention, and do that problem as a wrap-up.


\subsection*{Notes}
		\begin{enumerate}
			\item To talk about representations in a basis we technically need \emph{ordered} sets. In this class,
				when writing a set down and declaring it a basis, we imply that the order of the basis vectors
				is the left-to-right order that they're written in. The student's won't think about regular sets lacking
				order,
				so don't bring it up unless they ask.
			\item We use the notation $[\vec v]_{\mathcal X}$ to mean the list of numbers representing
				$\vec v$ in the basis $\mathcal X=\{\vec x_1,\vec x_2,\ldots\}$. We use $\mat{1\\2\\\vdots}_{\mathcal X}$ to
				mean the linear combination $1\vec x_1+2\vec x_2+\cdots$. Thus $[[\vec v]_{\mathcal X}]_{\mathcal X}=\vec v$.
			\item Lists of number that are not subscripted and are treated as vectors have an implicit
				subscript of $\mathcal E$, the standard basis. Students may be uncomfortable
				that sometimes we demand subscripts and sometimes we don't. It should always be clear from context
				whether we are talking about a list of numbers or a vector written in the standard basis, and on
				tests we will be very clear.
			\item For 2(c), make sure that the shortcut comes out that $[7\vec v]_{\mathcal X} = 7[\vec v]_{\mathcal X}$ and,
				in fact, changing basis is a linear operation.
			\item Problems 2(d) and 4 ask for value judgments. There's no ``right'' answer to these questions, but 
				some answers are better reasoned than others. Hold the students accountable for coming
				up with good reasons.
			\item No one will get to \#4, but if they do, encourage them to use a calculator to speed things along.
		\end{enumerate}
	\end{instructions}

\end{document}
