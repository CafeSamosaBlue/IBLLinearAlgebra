\documentclass[11pt]{exam}
\usepackage{amsmath}
\usepackage{amssymb}
\usepackage{graphicx}
\usepackage{enumitem}
\usepackage{amsfonts}
\usepackage{amssymb}
\usepackage{xparse}
\usepackage{ifthen}
\usepackage{geometry}
\noprintanswers

\newcommand {\DS} [1] {${\displaystyle #1}$}
\newcommand{\answer}[1]{{\bf Answer:} \; #1}
\newcommand{\vv}{\vspace{.2cm}}
\newcommand{\vvv}{\vspace{6cm}}

\newcommand{\ul}{$\underline{\phantom{xxx}}$}
\newcommand{\ull}{\underline{\phantom{xxx}}}
\newcommand{\xh}{\hat{\bf x}}
\newcommand{\yh}{\hat{\bf y}}
\newcommand{\zh}{\hat{\bf z}}
\newcommand{\R}{\mathbb{R}}
\newcommand{\C}{\mathbb{C}}
\newcommand{\Z}{\mathbb{Z}}
\newcommand{\N}{\mathbb{N}}
\newcommand{\proj}{\mathrm{proj}}
\renewcommand{\span}{\mathrm{span}}
\newcommand{\mat}[1]{\begin{bmatrix}#1\end{bmatrix}}
\newcommand{\floor}[1]{\lfloor #1 \rfloor}

\pagestyle{empty}


%%%%%%%%%%%%%%%%%%%%%%%%%%%%%%%%%%%%%%%%%
%  Edit course information here
%%%%%%%%%%%%%%%%%%%%%%%%%%%%%%%%%%%%%%%%%

\newcommand{\mthCourse}{MATH 110}
\newcommand{\mthTerm}{Fall 2013}
\newcommand{\mthTutorialNumber}{4}
\newcommand{\mthDate}{October 2, 2013}


%%%%%%%%%%%%%%%%%%%%%%%%%%%%%%%%%%%%%%%%%
\topmargin -1in
\textheight 10in

\begin{document}


%%%%%%%%%%%%%%%%%%%%%%%%%%%%%%%%%%%%%%%%%%%%%%%
% Main Questions
%%%%%%%%%%%%%%%%%%%%%%%%%%%%%%%%%%%%%%%%%%%%%%%
{\large
	\begin{center}
		{\bf \mthCourse, \mthTerm}\\ 
		{\bf Tutorial \#\mthTutorialNumber}\\
		{\bf \mthDate}
	\end{center}
}

\section*{Today's main problems}

Consider the vectors 
\[
	\vec a = \mat{1\\2\\1\\0}\qquad
	\vec b = \mat{-1\\-1\\0\\-1}\qquad
	\vec c = \mat{3\\4\\1\\2}\qquad
	\vec d = \mat{1\\3\\2\\-1}\qquad
\]

\begin{enumerate}
	\item Is the set $\{\vec a,\vec b,\vec c,\vec d\}$ linearly independent?
	\item Describe $\span\{\vec a,\vec b,\vec c,\vec d\}$ as a point, line,
		plane, or hyperplane and give its formula in vector form.
	\item If $A=[\vec a|\vec b|\vec c|\vec d]$ is the matrix whose columns are 
		the vectors $\vec a,\vec b,\vec c,\vec d$, what is the rank of $A$?
	\item You, while stuck in your moon base, discover that you urgently
		need to transmit the vectors $\vec a,\vec b,\vec c,\vec d$
		back to earth.  However, energy is very limited.  It takes
		100 units of energy to transmit the coordinates of a single
		vector.  It takes only 1 unit of energy to transmit a
		linear combination (e.g. $\vec x=7\vec y-2\vec z+\vec w$).
		What is the minimum amount of energy required to transmit 
		the vectors $\vec a,\vec b,\vec c,\vec d$ back to earth?
	
\end{enumerate}
\subsection*{Further Questions}
	Consider the plane $\mathcal P$ given by the equation
	\[
		2x-y+z=0.
	\]
\begin{enumerate}[resume]
	\item Find 
	vectors $\vec u$ and $\vec v$ so that $\mathcal{P}=\span\{\vec u,\vec v\}$.

	\item Can you find vectors $\vec u,\vec v,\vec w$ so that $\mathcal{P}=\span\{\vec u,
	\vec v,\vec w\}$?  If you can, find such vectors.  Otherwise, explain why it
		cannot be done.
\end{enumerate}




%%%%%%%%%%%%%%%%%%%%%%%%%%%%%%%%%%%%%%%%%%%%%%%
% Challenge questions
%%%%%%%%%%%%%%%%%%%%%%%%%%%%%%%%%%%%%%%%%%%%%%%
\newpage
{
	\begin{center}
		{\bf \mthCourse, \mthTerm}\\ 
		{\bf Tutorial \#\mthTutorialNumber}\\
		{\bf \mthDate}
	\end{center}
}

\section*{Challenge questions}
	Consider the plane $\mathcal P'$ given by the equation
	\[
		2x-y+z=4.
	\]
\begin{enumerate}[resume]
	\item Can you find
	vectors $\vec u$ and $\vec v$ so that $\mathcal{P}'=\span\{\vec u,\vec v\}$?
	\item Describe $\span \, \mathcal P'$.
	
	\item Recall the vectors $\vec a,\vec b,\vec c,\vec d$ from before.  
	Again you are stuck on a moon base and need to transmit the vectors
	$\vec a,\vec b,\vec c,\vec d$ to earth, but you've come up with a new
	encoding scheme.  With this new scheme it costs 50 units of energy
	to transmit the components of a vector.  Further, if you transmit
	the linear combination $\vec x=t\vec y+r\vec y+q\vec w$ it
	takes $|t|+|r|+|q|$ units of energy.  What is the least amount of energy
	you can use to transmit the vectors to earth?
\end{enumerate}



%%%%%%%%%%%%%%%%%%%%%%%%%%%%%%%%%%%%%%%%%%%%%%%
% TA instructions
%%%%%%%%%%%%%%%%%%%%%%%%%%%%%%%%%%%%%%%%%%%%%%%
\newpage
{\small
	\begin{center}
		{\bf \mthCourse, \mthTerm}\\ 
		{\bf Tutorial \#\mthTutorialNumber. Instructions for TAs}
	\end{center}
}

\subsection*{Objectives}

	Span and linear independence are the first abstract concepts 
	we encounter in this course and as such, we'd like to see
	how they fit with the concrete algorithms and equations
	we've been learning.  More specifically, the goal is to see
	how span, linear independence, and reduced row echelon form
	of a matrix all relate to each other.

\subsection*{Hidden objectives}

	Span and linear independence are easy once you understand them, 
	but it takes a lot experience working problems to understand the ins and outs.

\subsection*{Suggestions}
	
	Start by asking the class what the definition of span is and then what
	the definition of linear independence is.  Putting these definitions
	on the board should help people out.

\subsection*{Wrapup}
	Choose a question that most of the class has started but not yet finished,
	or a question that people particularly struggled with.

\subsection*{Solutions}
\begin{enumerate}
	\item[9.] 101.2 units.
\end{enumerate}
	

\end{document}
