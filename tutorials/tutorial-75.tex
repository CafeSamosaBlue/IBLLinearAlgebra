\documentclass[red]{tutorial}
\usepackage[no-math]{fontspec}
\usepackage{xpatch}
	\renewcommand{\ttdefault}{ul9}
	\xpatchcmd{\ttfamily}{\selectfont}{\fontencoding{T1}\selectfont}{}{}
	\DeclareTextCommand{\nobreakspace}{T1}{\leavevmode\nobreak\ }
\usepackage{polyglossia} % English please
	\setdefaultlanguage[variant=us]{english}
%\usepackage[charter,cal=cmcal]{mathdesign} %different font
%\usepackage{avant}
\usepackage{microtype} % Less badboxes


\usepackage[charter,cal=cmcal]{mathdesign} %different font
%\usepackage{euler}

\usepackage{blindtext}
\usepackage{calc, ifthen, xparse, xspace}
\usepackage{makeidx}
\usepackage[hidelinks, urlcolor=blue]{hyperref}   % Internal hyperlinks
\usepackage{mathtools} % replaces amsmath
\usepackage{bbm} %lower case blackboard font
\usepackage{amsthm, bm}
\usepackage{thmtools} % be able to repeat a theorem
\usepackage{thm-restate}
\usepackage{graphicx}
\usepackage{xcolor}
\usepackage{multicol}
\usepackage{fnpct} % fancy footnote spacing


\newcommand{\xh}{{{\mathbf e}_1}}
\newcommand{\yh}{{{\mathbf e}_2}}
\newcommand{\zh}{{{\mathbf e}_3}}
\newcommand{\R}{\mathbb{R}}
\newcommand{\Z}{\mathbb{Z}}
\newcommand{\N}{\mathbb{N}}
\newcommand{\proj}{\mathrm{proj}}
\newcommand{\Proj}{\mathrm{proj}}
\newcommand{\Comp}{\mathrm{comp}}
\newcommand{\Perp}{\mathrm{perp}}
\renewcommand{\span}{\mathrm{span}\,}
\newcommand{\Span}{\mathrm{span}\,}
\newcommand{\Img}{\mathrm{img}\,}
\newcommand{\Null}{\mathrm{null}\,}
\newcommand{\Range}{\mathrm{range}\,}
\newcommand{\rref}{\mathrm{rref}}
\newcommand{\rank}{\mathrm{rank}}
\newcommand{\Rank}{\mathrm{rank}}
\newcommand{\nnul}{\mathrm{nullity}}
\newcommand{\mat}[1]{\begin{bmatrix}#1\end{bmatrix}}
\newcommand{\chr}{\mathrm{char}}
\renewcommand{\d}{\mathrm{d}}


\theoremstyle{definition}
\newtheorem{example}{Example}[section]
\newtheorem{defn}{Definition}[section]

\theoremstyle{theorem}
\newtheorem{thm}{Theorem}[section]

\pgfkeys{/tutorial,
	name={Tutorial 7.5},
	author={Jason Siefken},
	course={MAT 223},
	date={March 17--23},
	term={Spring 2019},
	title={Constructing Examples}
	}

\begin{document}
	\begin{tutorial}

		\begin{objectives}
	In this tutorial you will be constructing matrices and linear transformations that satisfy
			given conditions, or explaining why they don't exist.

	These problems relate to the following course learning objectives:
			\textit{Use matrices to solve problems},
			\textit{translate between algebraic and geometric viewpoints to solve problems}, and
			\textit{clearly and correctly express the mathematical ideas of linear algebra to others}.
		\end{objectives}


\begin{enumerate}
	\item Write mathematically precise definitions of the rank of a matrix $A$ and the rank of a linear transformation $\mathcal{T}$.

	\item Give an example of a $2\times 3$ matrix $A$ with the specified rank, or explain why it cannot exist.
	\begin{enumerate}
		\item $\rank(A) = 1$
		\item $\rank(A) = 2$
		\item $\rank(A) = 3$
		\item $\rank(A) = 0$
	\end{enumerate}
		
	\item For a linear transformation $\mathcal L: \R^n\to \R^m$, explain how $\Rank(\mathcal L)$ relates to $m$ or $n$ under the following conditions:
	\begin{enumerate}
		\item $\mathcal L$ is one-to-one.
		\item $\mathcal L$ is {\bf not} one-to-one.
		\item $\mathcal L$ is onto.
		\item $\mathcal L$ is {\bf not} onto.
	\end{enumerate}
		
	\item Give examples of linear transformations $\mathcal T, \mathcal S: \R^3\to\R^3$ that satisfy the following, or explain why they cannot exist.
	\begin{enumerate}
		\item $\rank(\mathcal T)=\rank(\mathcal S)=\rank(\mathcal S\circ \mathcal T)=2$	
		\item $\rank(\mathcal T)=\rank(\mathcal S)=2$, and $\rank(\mathcal S\circ \mathcal T)=1$
		\item $\rank(\mathcal T)=\rank(\mathcal S)=2$, and $\rank(\mathcal S\circ \mathcal T)=3$
		\item $\rank(\mathcal T)=2$, $\rank(\mathcal S)=1$ and $\rank(\mathcal S\circ \mathcal T)=0$
	\end{enumerate}

	\item Tommy has returned once again and is working on similar matrices.
	\begin{enumerate}
		\item Write down a mathematically precise definition for two matrices to be similar.

\item
Tommy began with a $3\times 3$ matrix $A$ and multiplied by a change of basis matrix $X$ to find $B=XAX^{-1}$. His matrix computations gave
			\[A=\mat{
1 & 0 & 0  \\
0 & 2 & 0 \\
0 & 0 & 3
}\quad \text{ and } \quad
			B=\mat{
1 & 1 & 0  \\
2 & -1 & 0 \\
0 & 0 & 0
}.
\]
	Unfortunately, Tommy lost his paper containing $X$.
			Can you help him by finding a change of basis matrix
			$X$ that gives this solution or explaining to Tommy why no such matrix exists?
		\end{enumerate}
\end{enumerate}
	\end{tutorial}

	\begin{solutions}
		\begin{enumerate}
			\item
					The rank of a matrix $A$ is the number of pivots in rref($A$);
					the rank of a linear transformation $\mathcal T$ is the dimension of the range of $\mathcal T$.
			\item
			\begin{enumerate}
				\item $A=\begin{bmatrix}
					1 & 0 & 0 \\
					0 & 0 & 0
					\end{bmatrix}$ is one example. There are many others with linearly dependent rows, not both zero.
				\item $A=\begin{bmatrix}
					1 & 0 & 0 \\
					0 & 1 & 0
					\end{bmatrix}$ is one example. There are many others with linearly independent rows.
				\item Such a matrix does not exist. $A$ only has two rows, so rref($A$) can have at most two pivot positions. Equivalently, the columnspace of $A$ is a subspace of $\R^2$, so it can't be $3$ dimensional.
				\item $A=\begin{bmatrix}
					0 & 0 & 0 \\
					0 & 0 & 0
					\end{bmatrix}$ is the only example. There are no others.
			\end{enumerate}
			\item A linear transformation $\mathcal L:\R^n\to\R^m$
				is one-to-one if its nullity is $0$, and it is onto if its range is
				all of $\R^m$.
			\begin{enumerate}
				\item Using the rank-nullity theorem, we have rank$(\mathcal L) = n$.
				\item Similarly, if $\mathcal L$ is not one-to-one, we have rank$(\mathcal L) < n$. Note that we can never have rank$(\mathcal L)>n$ (why?).
				\item $\mathcal L$ is onto if rank$(\mathcal L)=m$.
				\item $\mathcal L$ is not onto if rank$(\mathcal L)<m$.
			\end{enumerate}
			\item
			\begin{enumerate}
				\item Since the ranks of $\mathcal T$ and $\mathcal S$ are both $2$,
					their ranges are both planes in $\R^{3}$. Let $\mathcal T=\mathcal
					S$ be projection onto the $xy$-plane. Then,
					$\mathcal S=\mathcal S\circ \mathcal T$ is rank $2$.

				\item Again, the ranges of $\mathcal T$ and $\mathcal S$ must be planes,
					but this time we want the range of $\mathcal S \circ \mathcal T$
					to be a line. Let $\mathcal T$ be projection onto the $xy$-plane
					and let $\mathcal S$ be projection onto the $xz$-plane. Then
					$\mathcal S\circ \mathcal T$ is projection onto the $x$-axis, which
					has rank $1$.

				\item This is impossible. Since $\Range(\mathcal S\circ \mathcal T)
					\subseteq$ $\Range(\mathcal S)$, we cannot have \[3=\Rank(\mathcal S\circ \mathcal T)
					\ >\ \Rank(\mathcal S)=2.\]

				\item The range of $\mathcal T$ is a plane, while the range of $\mathcal
					S$ is a line, and we would like $\mathcal S \circ \mathcal T$ to
					send every vector to $0$. Let $\mathcal S$ be projection onto the $xy$-plane
					and let $\mathcal T$ be projection onto the $z$-axis. Then
					$\Null(\mathcal S)=\Range(\mathcal T)$ and so $\mathcal S\circ\mathcal T=\mathbf 0$,
					the transformation which sends every vector to zero.
			\end{enumerate}
			\item Sorry Tommy, there must be a mistake somewhere in your lost notes.
				Comparing ranks, $\Rank(A)=3$ and $\Rank(B)=2$.
				Since $X$ is invertible, it does not change the
				dimension of the image of any subspace,
				so for similar matrices $A\sim B$, we must have $\Rank(A)=\Rank(B)$.		
			
		\end{enumerate}
	\end{solutions}
	\begin{instructions}
\subsection*{Learning Objectives}
	Students need to be able to\ldots
	\begin{itemize}
		\item Define rank (algebraically for matrices and geometrically for linear transformations)
		\item Produce examples of linear transformations
		\item Describe connection between matrix multiplication and composition of functions.
		\item Compute the image of a set of vectors under a transformation.
	\end{itemize}

\subsection*{Context}


\subsection*{What to Do}
  	\begin{itemize}
		\item Give the students a couple of minutes to think about
		problem 1. Then invite four of them up to the board. Have
		two define ``rank''
			of a matrix and two define ``rank'' of a linear
			transformation. Discuss any issues that their
			definitions may have. Make sure to end this
			portion with correct, clear definitions on
			the board. And make sure that the students
			understand them.

		\item Next, arrange the students in groups and give them
		10-12 minutes to work on problem 2 and 4. Be sure to
		walk around the room and try to get the students engaged
		and working. Once time is up, spend a few minutes at
		the board addressing any glaring issues you might've
		noticed while you were circling the room.

		\item Repeat the same process with either problem 3 or
		5. Ask the students to see if they prefer one problem
		over the other. Wrap-up the tutorial by discussing any
		issues you noticed while circling the room.
	\end{itemize}

\subsection*{Notes}
  	\begin{itemize}
		\item Be mindful of late-comers. During group work, try to approach them and make sure they are aware of what is going on, and that they are engaged and working in a group.
		\item Try to keep an organized board. A disorganized and chaotic board is just as annoying to the students as a jumbled mess on a test paper that you're trying to grade is to you.
	\end{itemize}
	\end{instructions}

\end{document}
