\documentclass[red]{tutorial}
\usepackage[no-math]{fontspec}
\usepackage{xpatch}
	\renewcommand{\ttdefault}{ul9}
	\xpatchcmd{\ttfamily}{\selectfont}{\fontencoding{T1}\selectfont}{}{}
	\DeclareTextCommand{\nobreakspace}{T1}{\leavevmode\nobreak\ }
\usepackage{polyglossia} % English please
	\setdefaultlanguage[variant=us]{english}
%\usepackage[charter,cal=cmcal]{mathdesign} %different font
%\usepackage{avant}
\usepackage{microtype} % Less badboxes


\usepackage[charter,cal=cmcal]{mathdesign} %different font
%\usepackage{euler}
 
\usepackage{blindtext}
\usepackage{calc, ifthen, xparse, xspace}
\usepackage{makeidx}
\usepackage[hidelinks, urlcolor=blue]{hyperref}   % Internal hyperlinks
\usepackage{mathtools} % replaces amsmath
\usepackage{bbm} %lower case blackboard font
\usepackage{amsthm, bm}
\usepackage{thmtools} % be able to repeat a theorem
\usepackage{thm-restate}
\usepackage{graphicx}
\usepackage{xcolor}
\usepackage{multicol}
\usepackage{fnpct} % fancy footnote spacing

 
\newcommand{\xh}{{{\mathbf e}_1}}
\newcommand{\yh}{{{\mathbf e}_2}}
\newcommand{\zh}{{{\mathbf e}_3}}
\newcommand{\R}{\mathbb{R}}
\newcommand{\Z}{\mathbb{Z}}
\newcommand{\N}{\mathbb{N}}
\newcommand{\proj}{\mathrm{proj}}
\newcommand{\Proj}{\mathrm{proj}}
\newcommand{\Perp}{\mathrm{perp}}
\renewcommand{\span}{\mathrm{span}\,}
\newcommand{\Span}{\mathrm{span}\,}
\newcommand{\Img}{\mathrm{img}\,}
\newcommand{\Null}{\mathrm{null}\,}
\newcommand{\Range}{\mathrm{range}\,}
\newcommand{\Kern}{\mathrm{kern}\,}
\newcommand{\rref}{\mathrm{rref}}
\newcommand{\rank}{\mathrm{rank}}
\newcommand{\Rank}{\mathrm{rank}}
\newcommand{\nnul}{\mathrm{nullity}}
\newcommand{\mat}[1]{\begin{bmatrix}#1\end{bmatrix}}
\newcommand{\chr}{\mathrm{char}}
\newcommand{\Dim}{\mathrm{dim}}
\renewcommand{\d}{\mathrm{d}}


\theoremstyle{definition}
\newtheorem{example}{Example}[section]
\newtheorem{defn}{Definition}[section]

\theoremstyle{theorem}
\newtheorem{thm}{Theorem}[section]

\pgfkeys{/tutorial,
	name={Tutorial 7},
	author={Jason Siefken},
	course={MAT 223},
	date={November 11},
	term={Fall 2019},
	title={Inverses}
	}

\begin{document}
	\begin{tutorial}

		\begin{objectives}
	In this tutorial you will be working with inverses and examining how they connect algebra
			and geometry.

	These problems relate to the following course learning objectives:
			\textit{Translate between algebraic and geometric viewpoints to solve problems}, and
	\textit{clearly and correctly express the mathematical ideas of linear algebra to others, 
	and understand and apply logical arguments and definitions that have been written by others}.
		\end{objectives}


\subsection*{Problems}

	\begin{enumerate}
	\item Let $f:\R^n\to\R^n$ be a function and let $A$ be a matrix. Write down the definition of (i) what it means
		for $f$ to be \emph{invertible} and (ii) what it means for $A$ to be invertible.
	
	\item Below are several samples from student answers to a MAT223 exam question relating to solving the matrix equation
		$A\vec x=\vec b$. For each sample, decide whether it is totally correct, mostly correct, mostly incorrect, or 
		totally incorrect.
		If it is not totally correct, explain what is wrong, and if possible, how to fix it.

		You may assume the student can correctly compute $A^{-1}$.
		\begin{enumerate}
			\item $A\vec x=\vec b$ $\quad\implies\quad$ $\vec x=\vec bA^{-1}$.
			\item $A\vec x=\vec b$ $\quad\implies\quad$ $\vec x=A^{-1}\vec b$.
			\item $A\vec x=\vec b$ $\quad\implies\quad$ $\vec x=\vec b/A$.
			\item $A\vec x=\vec b$ $\quad\implies\quad$ $\vec x=\tfrac{1}{A}\vec b$.
			\item $A\vec x=\vec b$ $\quad\implies\quad$ $A/\vec b=1/\vec x$. $\quad\implies\quad$ $1/(A/\vec b)=\vec x$.
		\end{enumerate}

	\item Let $R:\R^2\to\R^2$ be rotation counter-clockwise by $30^\circ$; let $D:\R^2\to\R^2$ be reflection across the
		line $y=4x$; let $P:\R^2\to\R^2$ be projection onto the line $y=4x$; and let $S:\R^2\to\R^2$ be the transformation
			that doubles the length of every vector.

			For each transformation, (i) decide if it is invertible, and (ii) describe the inverse-transformation in words.
		\item Let $R$, $D$, $P$, and $S$ be defined as before.
			\begin{enumerate}
				\item Determine the rank of $R$, $D$, $P$, and $S$.
				\item Does rank relate to invertibility? Write down an argument to support your conjecture.
				\item Find matrices for $R$, $D$, $P$, and $S$. How does the invertibility of these
					matrices relate to the invertibility of the original transformations?
					Is this relationship affected by which basis you choose to represent the
					transformation in?
			\end{enumerate}
	\end{enumerate}



	\end{tutorial}


	\begin{solutions}
		\begin{enumerate}
			\item $f:\R^n\to\R^n$ is invertible if there exists a function $g:\R^n\to\R^n$ so
				that $f\circ g=g\circ f=$id, the identity function. 
				The matrix $A$ is invertible if there exists a matrix $B$ so that
				$AB=I$ and $BA=I$.
			\item \begin{enumerate}
					\item Totally incorrect. If $\vec b$ is a column
						vector $\vec b A^{-1}$ won't be defined.
						It should be $\vec x=A^{-1}\vec b$.
					\item Totally correct.
					\item Totally incorrect. There is no division operation
						for matrices. Instead, you must multiply both sides by $A^{-1}$.
					\item Mostly incorrect. The notation $\tfrac{1}{A}$ is not defined. We
						use the notation $A^{-1}$ for the matrix such that $A^{-1}A=AA^{-1}=I$.
					\item Totally incorrect. You cannot divide by a vector.
			\end{enumerate}
			\item \begin{itemize}
					\item[$R$] This transformation is invertible and its inverse is
						clockwise rotation by $30^{\circ}$.
					\item[$D$] This transformation is invertible and its inverse is
						itself.
					\item[$P$] This transformation is not invertible. Since 
						$P(\vec 0) = P\left(\mat{-4\\1}\right)=\vec 0$, there
						is no way to ``undo'' $P$. Formally, $P$ is not one-to-one,
						so it cannot be invertible.
					\item[$S$] This transformation is invertible and its inverse is
						halving the length of each vector.
			\end{itemize}
			\item \begin{enumerate}
				\item $R,D,S$ are rank 2 and $P$ is rank 1.
				\item Yes. If a transformation from $\R^n\to\R^n$ is invertible 
					its rank must be $n$. We can argue as follows:

					Suppose $T:\R^n\to\R^n$ has rank $< n$. Then the range of $T$ cannot
					be all of $\R^n$. Thus, there cannot exist a transformation $S$ so that
					$T\circ S=$id, where id is the identity function on all of $\R^n$.
				\item 
					\[
						[R]_{\mathcal E} = \mat{\sqrt{3}/2&-1/2\\1/2&\sqrt{3}/2}\qquad
						[D]_{\mathcal E} = \tfrac{1}{17}\mat{-15&8\\8&15}
						\]\[
						[P]_{\mathcal E} = \tfrac{1}{17}\mat{1&4\\4&16}\qquad
						[S]_{\mathcal E} = \mat{2&0\\0&2}
					\]
					The matrix for a transformation is invertible if and only if the
					transformation is invertible. This is not affected by the choice of basis.
			\end{enumerate}
		\end{enumerate}
	\end{solutions}
	\begin{instructions}


\subsection*{Learning Objectives}
	Students need to be able to\ldots
	\begin{itemize}
		\item Read others' mathematical arguments and identify whether they are correct
		\item Multiply matrices in the correct order when solving matrix equations
		\item Apply the meaning of invertibility (that a transformation can be ``undone'')
		\item Formally define invertibility
		\item Recognize the identity function as a legitimate function
	\end{itemize}

\subsection*{Context}
	In lecture, students have seen rank of a matrix and a linear transformation; null space of a matrix/linear
		transformation;	column space, row space of a matrix; and elementary matrices. However, most
		of them will be shaky on what an inverse actually ``means'' and even more shaky on how to
		define the inverse of a function (even though this is grade 11 material). Many lectures,
		but not all, have used the terms ``one-to-one'' and ``onto'' when talking about inverses.
		You may use these terms (the students should know them from high school). Prefer ``one-to-one''
		and ``onto'' over ``injective'' and ``surjective''.

\subsection*{What to Do}
	Start the tutorial by stating the day's learning objectives and that today has two focuses: \emph{inverses}
		and \emph{reading others' work}. You may tell them that the idea of inverse \emph{functions} is used
		in general contexts, not just in linear algebra, but that we only
		deal with inverse \emph{matrices} in linear algebra. This is why they are asked for two
		definitions to start.

		After everyone has written down a definition for inverse functions and inverse matrices, 
		get a correct definition on the board and have a short ($\leq $ 3 minute) discussion on why
		inverses are useful for solving equations/matrix equations. Then, have them start \#2.

		\#2 shouldn't take long, but should allow a for a good discussion of the fact that
		when we write out matrix equations, they look the same as equations with numbers, but if
		we don't remember that they're actually linear algebra objects, we'll do nonsensical things.
		\emph{The power of algebra (in the symbolic sense) is that it allows us to shut off our brains
		and follow symbolic steps, but we cannot shut off our brains too much!}

		\#3 asks students to connect geometric transformations with
		the formal definition of a function being invertible. This will be hard for some and easy for
		others. If they get stuck and frustrated, you can ask them to try 4(c) and then come back to 3.

		6 minutes before the end of class, pick a suitable problem to do as a wrap-up.

\subsection*{Notes}
	\begin{itemize}
		\item Students will be uncomfortable with function composition. They would prefer
			to write $h(x) = f(g(x))$ over $h=f\circ g$. We want them to get comfortable
			with $f\circ g$ notation, but you may use $f(g(x))$ notation if they're confused.
		\item \#2 is trivial in some sense. And, we could extend notational definitions to
			make many of those correct. However, in this class we're using the standard 
			North-American notations, so only (b) is acceptable.
		\item For \#3 encourage them to draw some ``before and after'' pictures of vectors
			subjected to each transformation. This will aid their thinking (and should
			be their natural inclination).
		\item For \#4(b), it is easier to argue invertible$\implies$full rank than 
			full rank$\implies$invertible. Focus on invertible$\implies$full rank and
			leave the converse as a challenge for the stronger students.

			Also, you may use matrix theory when arguing \#4, but make sure
			that a distinction between \emph{matrices} and \emph{transformations} is made.
			Just like we can tell things about vectors by looking
			at their representation in a basis, so can we tell things about transformations
			by looking at their matrix representation. However these aren't the same thing!
	\end{itemize}

	\end{instructions}

\end{document}
