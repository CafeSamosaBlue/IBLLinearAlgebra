\documentclass[11pt]{exam}
\usepackage{amsmath}
\usepackage{amssymb}
\usepackage{graphicx}
\usepackage{enumitem}
\usepackage{amsfonts}
\usepackage{amssymb}
\usepackage{xparse}
\usepackage{ifthen}
\usepackage{geometry}
\noprintanswers

\newcommand {\DS} [1] {${\displaystyle #1}$}
\newcommand{\answer}[1]{{\bf Answer:} \; #1}
\newcommand{\vv}{\vspace{.2cm}}
\newcommand{\vvv}{\vspace{6cm}}

\newcommand{\ul}{$\underline{\phantom{xxx}}$}
\newcommand{\ull}{\underline{\phantom{xxx}}}
\newcommand{\xh}{\hat{\bf x}}
\newcommand{\yh}{\hat{\bf y}}
\newcommand{\zh}{\hat{\bf z}}
\newcommand{\R}{\mathbb{R}}
\newcommand{\C}{\mathbb{C}}
\newcommand{\Z}{\mathbb{Z}}
\newcommand{\N}{\mathbb{N}}
\newcommand{\proj}{\mathrm{proj}}
\newcommand{\mat}[1]{\begin{bmatrix}#1\end{bmatrix}}
\newcommand{\floor}[1]{\lfloor #1 \rfloor}

\renewcommand{\span}{\mathrm{span}\,}
\newcommand{\rref}{\mathrm{rref}}
\newcommand{\rank}{\mathrm{rank}}
\newcommand{\nnul}{\mathrm{nullity}}

\pagestyle{empty}


%%%%%%%%%%%%%%%%%%%%%%%%%%%%%%%%%%%%%%%%%
%  Edit course information here
%%%%%%%%%%%%%%%%%%%%%%%%%%%%%%%%%%%%%%%%%

\newcommand{\mthCourse}{MATH 110}
\newcommand{\mthTerm}{Fall 2013}
\newcommand{\mthTutorialNumber}{11}
\newcommand{\mthDate}{November 27, 2013}


%%%%%%%%%%%%%%%%%%%%%%%%%%%%%%%%%%%%%%%%%
\topmargin -1in
\textheight 10in

\begin{document}


%%%%%%%%%%%%%%%%%%%%%%%%%%%%%%%%%%%%%%%%%%%%%%%
% Main Questions
%%%%%%%%%%%%%%%%%%%%%%%%%%%%%%%%%%%%%%%%%%%%%%%
{\large
	\begin{center}
		{\bf \mthCourse, \mthTerm}\\ 
		{\bf Tutorial \#\mthTutorialNumber}\\
		{\bf \mthDate}
	\end{center}
}

\section*{Today's main problems}
		$\mathcal V=\{\vec v_1,\vec v_2,\vec v_3, \vec v_4\}$ where
		\[
			\vec v_1=\mat{0\\2\\1\\0}\qquad
			\vec v_2=\mat{1\\-1\\0\\0}\qquad
			\vec v_3=\mat{1\\2\\0\\-1}\qquad
			\vec v_4=\mat{1\\0\\0\\1}
		\]
\begin{enumerate}
	\item Apply the Gram-Schmidt process to $\mathcal V$ to find an orthonormal
		basis $\mathcal B=\{\vec b_1,\vec b_2,\vec b_3,\vec b_4\}$
		for $\R^4$.  Note that some of the vectors in $V$
		are already orthogonal to eachother.  You may reorder the vectors
		before you apply Gram-Schmidt to minimize the amount of work
		you need to do.

\end{enumerate}
\subsection*{Further Questions}
	Let $B=[\vec b_1|\vec b_2|\vec b_3]$ where $\vec b_i\in\mathcal B$ from problem 1.
	Define 
	\[
		C=\span\{\vec b_1,\vec b_2,\vec b_3\}\qquad
		\vec c_1=\mat{1\\2\\3\\4}\qquad
		\vec c_2=\mat{2\\1\\1\\1}.
	\]
\begin{enumerate}[resume]
	
	\item 
	\begin{enumerate}
		\item Is the system $B\vec x=\vec c_1$ consistent?
		\item Is the system $B\vec x=\vec c_2$ consistent?
		\item Is the system $B\vec x=\proj_C\vec y$ consistent for any choice of $\vec y$?
	\end{enumerate}
	\item Even if a system $A\vec x=\vec b$ is inconsistent, we can attempt to find
		a best solution.  That is, we can attempt to find a vector $\vec x$ so
		that $A\vec x$ is as close to $\vec b$ as it can be while still
		being in the column space.  Find the best solution to the
		system $B\vec x=\vec c_1$ (also called the \emph{least squares}
		solution) by projecting $\vec c_1$ onto the column space of $B$ and then
		solving $B\vec x=\proj_C\vec c_1$.
\end{enumerate}




%%%%%%%%%%%%%%%%%%%%%%%%%%%%%%%%%%%%%%%%%%%%%%%
% Challenge questions
%%%%%%%%%%%%%%%%%%%%%%%%%%%%%%%%%%%%%%%%%%%%%%%
\newpage
{
	\begin{center}
		{\bf \mthCourse, \mthTerm}\\ 
		{\bf Tutorial \#\mthTutorialNumber}\\
		{\bf \mthDate}
	\end{center}
}

\section*{Challenge questions}

\begin{enumerate}[resume]
	\item Let $A$ and $B$ be orthogonal matrices.
	\begin{enumerate}
		\item Prove $A(A^T+B^T)B=A+B$.
		\item Use part (a) to prove that if $\det A+\det B=0$ then $A+B$
			is not invertible.
	\end{enumerate}
	\item We can define a dot product on polynomials.  If $p(x)$ and
		$q(x)$ are polynomials, then \[\displaystyle 
		p\cdot q=\int_{-1}^1 p(x)q(x)\,\mathrm{d} x.\]
		Use the Gram-Schmidt process on the basis $\mathcal P=\{1,x,x^2,x^3,x^4\}$
		to come up with an orthonormal basis for polynomials of degree at most 4.
		If you graph the polynomials that make up your basis, what do you notice?
		(These polynomials are called the Legendre polynomials.)

\end{enumerate}



%%%%%%%%%%%%%%%%%%%%%%%%%%%%%%%%%%%%%%%%%%%%%%%
% TA instructions
%%%%%%%%%%%%%%%%%%%%%%%%%%%%%%%%%%%%%%%%%%%%%%%
\newpage
{\small
	\begin{center}
		{\bf \mthCourse, \mthTerm}\\ 
		{\bf Tutorial \#\mthTutorialNumber. Instructions for TAs}
	\end{center}
}

\subsection*{Objectives}
	Students need practice doing Gram-Schmidt and especially doing
	careful and methodical arithmetic.


\subsection*{Hidden objectives}


\subsection*{Suggestions}
	Though they have seen Gram-Schmidt in class, most of them won't remember
	the procedure.  It is worth reminding them of the idea behind Gram-Schmidt
	at the beginning: ``take a set of vectors that is already orthogonal
	to one another (this could be a single vector) and add a new vector by subtracting
	off its projection onto each vector already in your set.''  If you can draw a 3-d picture
	to illustrate the idea, more power to you, but don't get too carried away.  They 
	seen this in class and your job isn't to give a lecture on the topic.

	If students are having computational trouble with Gram-Schmidt, make them 
	be methodical.  Make them get out a clean sheet of paper and write all the
	intermediate calculations down.  I know fractions are super hard, but 
	sometimes they're just there.  Also, have them resist the urge to normalize until
	they've orthogonalized everything.

\subsection*{Wrapup}
	The objective of this tutorial is question 1.  If people got question 1, but they
	struggled with the arithmetic and bookkeeping, it is worth you doing this methodically
	as a wrapup.  A mathematician's skill of being systematic, organized, and renaming
	things where appropriate isn't learned over night.  If everyone had no organizational
	trouble with the computations, you can consider doing a different problem as a wrapup.


\subsection*{Solutions}
\begin{enumerate}
	\item
\end{enumerate}
	

\end{document}
