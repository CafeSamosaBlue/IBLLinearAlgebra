\documentclass[red]{tutorial}
\usepackage[no-math]{fontspec}
\usepackage{xpatch}
	\renewcommand{\ttdefault}{ul9}
	\xpatchcmd{\ttfamily}{\selectfont}{\fontencoding{T1}\selectfont}{}{}
	\DeclareTextCommand{\nobreakspace}{T1}{\leavevmode\nobreak\ }
\usepackage{polyglossia} % English please
	\setdefaultlanguage[variant=us]{english}
%\usepackage[charter,cal=cmcal]{mathdesign} %different font
%\usepackage{avant}
\usepackage{microtype} % Less badboxes


\usepackage[charter,cal=cmcal]{mathdesign} %different font
%\usepackage{euler}

\usepackage{blindtext}
\usepackage{calc, ifthen, xparse, xspace}
\usepackage{makeidx}
\usepackage[hidelinks, urlcolor=blue]{hyperref}   % Internal hyperlinks
\usepackage{mathtools} % replaces amsmath
\usepackage{bbm} %lower case blackboard font
\usepackage{amsthm, bm}
\usepackage{thmtools} % be able to repeat a theorem
\usepackage{thm-restate}
\usepackage{graphicx}
\usepackage{xcolor}
\usepackage{multicol}
\usepackage{fnpct} % fancy footnote spacing


\newcommand{\xh}{{{\mathbf e}_1}}
\newcommand{\yh}{{{\mathbf e}_2}}
\newcommand{\zh}{{{\mathbf e}_3}}
\newcommand{\R}{\mathbb{R}}
\newcommand{\Z}{\mathbb{Z}}
\newcommand{\N}{\mathbb{N}}
\newcommand{\proj}{\mathrm{proj}}
\newcommand{\Proj}{\mathrm{proj}}
\newcommand{\Perp}{\mathrm{perp}}
\renewcommand{\span}{\mathrm{span}\,}
\newcommand{\Span}{\mathrm{span}\,}
\newcommand{\Img}{\mathrm{img}\,}
\newcommand{\Null}{\mathrm{null}\,}
\newcommand{\Range}{\mathrm{range}\,}
\newcommand{\rref}{\mathrm{rref}}
\newcommand{\rank}{\mathrm{rank}}
\newcommand{\Rank}{\mathrm{rank}}
\newcommand{\nnul}{\mathrm{nullity}}
\newcommand{\mat}[1]{\begin{bmatrix}#1\end{bmatrix}}
\newcommand{\chr}{\mathrm{char}}
\renewcommand{\d}{\mathrm{d}}


\theoremstyle{definition}
\newtheorem{example}{Example}[section]
\newtheorem{defn}{Definition}[section]

\theoremstyle{theorem}
\newtheorem{thm}{Theorem}[section]

\pgfkeys{/tutorial,
	name={Tutorial 10},
	author={Jason Siefken},
	course={MAT 223},
	date={March~31--April~5},
	term={Spring 2019},
	title={Eigenstuff}
	}

\begin{document}
	\begin{tutorial}

		\begin{objectives}
	In this tutorial you will be working with eigenvectors and eigenvalues.

	These problems relate to the following course learning objectives:
			\textit{Clearly and correctly express the mathematical ideas of linear algebra to others}
			and \textit{translate between algebraic and geometric viewpoints to solve problems}.
		\end{objectives}


\begin{enumerate}
	\item Let $T:\R^n\to\R^n$ be a linear transformation.
	\begin{enumerate}
		\item Write a mathematically precise definition of what it means for $\vec v$ to be an eigenvector
			for $T$.
		\item Describe in plain English what it means for $\vec v$ to be an eigenvector for $T$.
	\end{enumerate}
	\item Consider
		\[
			A=\mat{3&-2\\2&-2}\qquad
			B=\mat{5&-6\\3&-4}\qquad
			C=\mat{-3&3\\-3&3}
		\]
		\[
			\vec v_1=\mat{1\\2}\qquad
			\vec v_2=\mat{2\\1}\qquad
			\vec v_3=\mat{3\\4}\qquad
			\vec v_4=\mat{1\\1}\qquad
			\vec v_5=\mat{1\\0}\qquad
			\vec v_6=\mat{0\\0}
		\]
		For each $\vec v_i$, determine whether $\vec v_i$ is an eigenvector for $A$, $B$, or $C$. If so,
		identify the corresponding eigenvalue.
	\item Let $\mathcal P:\R^2\to\R^2$ be projection onto the $x$-axis and $\mathcal R:\R^2\to\R^2$ be rotation
		counter-clockwise by $90^\circ$.

		Find all eigenvectors and eigenvalues for $\mathcal P$ and $\mathcal R$.
	\item Suppose $T:\R^n\to\R^n$ is a linear transformation and that $\vec v$ is an eigenvector for
		$T$ with eigenvalue $2$. Is it possible that $7\vec v$ is an eigenvector for $T$ with eigenvalue
		$14$? Prove your answer.

	\item Let $T:\R^2\to\R^2$ be a linear transformation. $T$ has an eigenvector $\vec v_1$ with eigenvalue $1$
		and an eigenvector $\vec v_2$ with eigenvalue $1/2$. Further, $\|\vec v_1\|,\|\vec v_2\|\leq 100$.
		\begin{enumerate}
			\item Let $\vec w=\vec v_1+\vec v_2$. Approximate $T^{100}\vec w$.
			\item Is $\{\vec v_1,\vec v_2\}$ a basis for $\R^2$? Prove your answer.
			\item Suppose $\vec v_1=\mat{1\\2}$ and $\vec v_2=\mat{3\\5}$. With this information,
				can you approximate $T^{100}\mat{a\\b}$?
			\item Can you generalize your procedure from (c) to any linear transformation $S:\R^2\to\R^2$
				that has two positive, distinct eigenvalues?
		\end{enumerate}

\end{enumerate}
	\end{tutorial}

	\begin{solutions}
		\begin{enumerate}
			\item \begin{enumerate}
					\item $\vec v$ is an eigenvector for $T$ if $\vec v\neq \vec 0$ and
						$T\vec v=\lambda \vec v$ for some scalar $\lambda$.
					\item $\vec v$ is an eigenvector for $T$ if it doesn't change directions
						when $T$ is applied.
			\end{enumerate}
			\item
				For $A$:
				$\vec v_1$ is an eigenvector with eigenvalue $-1$;
				$\vec v_2$ is an eigenvector with eigenvalue $2$.

				For $B$:
				$\vec v_2$ is an eigenvector with eigenvalue $2$;
				$\vec v_4$ is an eigenvector with eigenvalue $-1$.

				For $C$:
				$\vec v_4$ is an eigenvector with eigenvalue $0$.
			\item $\mathcal P$ has eigenvalues of $0$ and $1$. All non-zero multiples of $\mat{0\\1}$
				have eigenvalue $0$ and non-zero multiples of $\mat{1\\0}$ have eigenvalue $1$.

				$\mathcal R$ has no real eigenvalues and no real eigenvectors. Every non-zero
				vector changes direction when $\mathcal R$ is applied.
			\item It is not possible.

				\emph{Proof:} Since $\vec v$ is an eigenvector for $T$ with eigenvalue
				$2$, we know $T\vec v=2\vec v$. Since $T$ is linear, we know
				\[
					T(7\vec v)=7T\vec v=14\vec v,
				\]
				and so $7\vec v$ is an eigenvector for $T$ with eigenvalue $2$. There is no
				other possibility.
			\item \begin{enumerate}
				\item $T^{100}\vec w=\vec v_1+\tfrac{1}{2^{100}}\vec v_2 \approx \vec v_1$.
				\item Yes. Since $\vec v_1$ and $\vec v_2$ are eigenvectors, neither is zero.
				Suppose $\{\vec v_1,\vec v_2\}$ is linearly dependent. Then, $\vec v_1=t\vec v_2$
					for some $t$ (because $\vec v_2\neq \vec 0$). Computing,
					\[
						T\vec v_1=T(t\vec v_2)=tT\vec v_2=\tfrac{t}{2}\vec v_2 = \tfrac{1}{2}\vec v_1,
					\]
					and so $\vec v_1$ would have an eigenvalue of $\tfrac{1}{2}$. But, this
					is a contradiction since the eigenvalue
					of $\vec v_1$ is $1$. Thus $\{\vec v_1,\vec v_2\}$ must be linearly independent.

					Finally, a linearly independent set of two vectors in $\R^2$ must span all of $\R^2$,
					and so $\{\vec v_1,\vec v_2\}$ is a basis for $\R^2$.
				\item Yes. First write $\mat{a\\b}$ in the $\{\vec v_1,\vec v_2\}$ basis. Then throw away
					the $\vec v_2$ component.

					The matrix that converts from the standard basis to the $\{\vec v_1,\vec v_2\}$ basis is
					\[
						C=\mat{-5&3\\2&-1}.
					\]
					The matrix that drops second coordinate of a vector is
					\[
						D=\mat{1&0\\0&0}.
					\]
					Thus
					\[
						T^{100}\mat{a\\b} \approx C^{-1}DC\mat{a\\b}=\mat{-5a+3b\\-10a+6b}.
					\]
				\item Yes. Suppose $S$ has a unique, largest, positive eigenvalue $\lambda$. Then $S'=\tfrac{1}{\lambda}S$
					behaves similarly to $T$. Approximate the same way, then multiply by $\lambda$.
			\end{enumerate}
		\end{enumerate}
	\end{solutions}
	\begin{instructions}
\subsection*{Learning Objectives}
	Students need to be able to\ldots
	\begin{itemize}
		\item Define eigenvector and eigenvalue
		\item Have an intuitive geometric understanding of eigenvectors
		\item Use the definition of eigenvector to classify vectors as eigenvectors or not
		\item Apply eigenvectors to solve problems
	\end{itemize}

\subsection*{Context}
	Students have been introduced to determinants, characteristic polynomials, and the definition
		of eigenvectors/values in lecture. Some students may have started diagonalization in
		lecture.

	In this course we will only pursue real eigenvalues/vectors. Some students may know complex
		numbers from high school, but most will not.


\subsection*{What to Do}
	Start the tutorial by stating the day's learning objectives. Emphasize that eigenvectors/values
		are the capstone of this course and they take a while to get used to. Today we are getting used
		to them.

		Have students start in groups on \#1. They should be used to the style of part (a) by now,
		but they will probably struggle with part (b). Try to get them to produce something better than
		reading the mathematical definition aloud.

		\#2 is a computational question that follows straight from the definition. If you like,
		divide the class into three teams.
		One team will check for $A$, another for $B$, and the last for $C$. Write up (or have each
		team write up) a summary of their results. Then have everyone verify each team's result.

		Before finishing \#2, make sure to have a discussion about $\vec v_6$. Is it an eigenvector?
		Does it go to a multiple of itself?

		It is likely that \#1 and \#2 will take most of class time. If you do have extra time, continue as usual,
		letting students work in small groups on \#3 and then having a mini-discussion when half the groups
		have figured it out.

		6 minutes before the end of class, pick a suitable problem to do as a wrap-up.


\subsection*{Notes}
	\begin{itemize}
		\item For \#1(b), many students will write ``$\vec v$ is an eigenvector for $T$ if
			$\vec v$ is non-zero and $T$ of $\vec v$ is $\lambda$ times $\vec v$ for some scalar
			$\lambda$.'' This is reading the mathematical definition aloud and is not what we want.
			We want students to rephrase $T\vec v=\lambda \vec v$ in terms of a vector not changing
			direction.
		\item For \#2, some students will get confused between $\vec v\neq \vec 0$ and $\lambda \neq 0$.
			$C,\vec v_4$ has eigenvalue $0$ precisely so this confusion will come up. Make sure to
			have a conversation about this whether or not it comes up naturally.
		\item Encourage students to draw pictures for \#3. They will naturally try to start by writing
			down a matrix for $\mathcal P$ and $\mathcal R$, but that short-circuits thinking.
		\item
		When talking about eigenvalues of a transformation $T$
		not existing, always make sure to say ``$T$ has no real eigenvalues''. Students who
		know about complex numbers can have a deeper conversation with you about them after class.
		\item In \#4, some might find it so ``obvious'' that they don't know how to prove it. That's largely
			the point. They should be able to prove obvious things!
		\item \#5 culminates in diagonalization, but it asks students to make a judgement about whether
			a vector is ``approximated''. The fact that $\|\vec v_1\|,\|\vec v_2\|\leq 100$ is essential
			for this, but isn't the point of the problem. The point is that there is a unique, largest, positive
			eigenvalue.

	\end{itemize}
	\end{instructions}

\end{document}
