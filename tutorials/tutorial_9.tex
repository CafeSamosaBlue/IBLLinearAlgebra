\documentclass[11pt]{exam}
\usepackage{amsmath}
\usepackage{amssymb}
\usepackage{graphicx}
\usepackage{enumitem}
\usepackage{amsfonts}
\usepackage{amssymb}
\usepackage{xparse}
\usepackage{ifthen}
\usepackage{geometry}
\noprintanswers

\newcommand {\DS} [1] {${\displaystyle #1}$}
\newcommand{\answer}[1]{{\bf Answer:} \; #1}
\newcommand{\vv}{\vspace{.2cm}}
\newcommand{\vvv}{\vspace{6cm}}

\newcommand{\ul}{$\underline{\phantom{xxx}}$}
\newcommand{\ull}{\underline{\phantom{xxx}}}
\newcommand{\xh}{\hat{\bf x}}
\newcommand{\yh}{\hat{\bf y}}
\newcommand{\zh}{\hat{\bf z}}
\newcommand{\R}{\mathbb{R}}
\newcommand{\C}{\mathbb{C}}
\newcommand{\Z}{\mathbb{Z}}
\newcommand{\N}{\mathbb{N}}
\newcommand{\proj}{\mathrm{proj}}
\newcommand{\mat}[1]{\begin{bmatrix}#1\end{bmatrix}}
\newcommand{\floor}[1]{\lfloor #1 \rfloor}

\renewcommand{\span}{\mathrm{span}\,}
\newcommand{\rref}{\mathrm{rref}}
\newcommand{\rank}{\mathrm{rank}}
\newcommand{\nnul}{\mathrm{nullity}}

\pagestyle{empty}


%%%%%%%%%%%%%%%%%%%%%%%%%%%%%%%%%%%%%%%%%
%  Edit course information here
%%%%%%%%%%%%%%%%%%%%%%%%%%%%%%%%%%%%%%%%%

\newcommand{\mthCourse}{MATH 110}
\newcommand{\mthTerm}{Fall 2013}
\newcommand{\mthTutorialNumber}{9}
\newcommand{\mthDate}{November 6, 2013}


%%%%%%%%%%%%%%%%%%%%%%%%%%%%%%%%%%%%%%%%%
\topmargin -1in
\textheight 10in

\begin{document}


%%%%%%%%%%%%%%%%%%%%%%%%%%%%%%%%%%%%%%%%%%%%%%%
% Main Questions
%%%%%%%%%%%%%%%%%%%%%%%%%%%%%%%%%%%%%%%%%%%%%%%
{\large
	\begin{center}
		{\bf \mthCourse, \mthTerm}\\ 
		{\bf Tutorial \#\mthTutorialNumber}\\
		{\bf \mthDate}
	\end{center}
}

\section*{Today's main problems}

		\[
			A=\mat{3&-2\\2&-2}\qquad
			B=\mat{5&-6\\3&-4}\qquad
			C=\mat{-3&4\\-3&5}
		\]
		\[
			\vec v_1=\mat{1\\2}\qquad
			\vec v_2=\mat{2\\1}\qquad
			\vec v_3=\mat{3\\4}\qquad
			\vec v_4=\mat{1\\1}\qquad
			\vec v_5=\mat{1\\0}
		\]
\begin{enumerate}
	\item For each $\vec v_i$, identify whether $\vec v_i$ is an eigenvector 
		of $A$, $B$, or $C$.  If so, find the corresponding eigenvalue.

	\item Find all eigenvectors and eigenvalues of $C$.

	\item $F=\mat{1&2&1\\-1&-1&3\\4&0&k}$
	\begin{enumerate}
		\item Compute $\det (F)$.
		\item For what values of $k$ is $F$ invertible?
		\item For what values of $k$ is zero an eigenvalue of $F$?
	\end{enumerate}
	

\end{enumerate}
\subsection*{Further Questions}
\begin{enumerate}[resume]
	
	\item $R=\mat{0&1\\-1&0}$
	\begin{enumerate}
		\item Explain what $R$ does geometrically.
		\item Compute the eigenvalues and eigenvectors of $R$.
		\item Explain geometrically why you should or shouldn't expect real eigenvectors
			or eigenvalues for $R$.  Can you generalize this result?
	\end{enumerate}
	\item 
	Let $E=\mat{2&1&0\\0&2&0\\0&0&3}$.
		\begin{enumerate}
			\item Find the eigenvectors and eigenvalues of $E$.
			\item List the algebraic and geometric multiplicity of each eigenvalue.
			\item Can you form a basis for $\R^3$ consisting of eigenvectors
				of $E$?
		\end{enumerate}
	\item For the matrix $F$, you know $\mat{1\\-1}$ is an eigenvector
		with eigenvalue $3$ and $\mat{2\\3}$ is an eigenvector with eigenvalue
		$2$.
		\begin{enumerate}
			\item Is there a basis for $\R^2$ consisting of eigenvectors of $F$?
			\item Write $\vec v=\mat{5\\5}$ in the eigenbasis.
			\item Compute $F\vec v$.
		\end{enumerate}
\end{enumerate}




%%%%%%%%%%%%%%%%%%%%%%%%%%%%%%%%%%%%%%%%%%%%%%%
% Challenge questions
%%%%%%%%%%%%%%%%%%%%%%%%%%%%%%%%%%%%%%%%%%%%%%%
\newpage
{
	\begin{center}
		{\bf \mthCourse, \mthTerm}\\ 
		{\bf Tutorial \#\mthTutorialNumber}\\
		{\bf \mthDate}
	\end{center}
}

\section*{Challenge questions}

\begin{enumerate}[resume]
	\item Find the matrix $F$ from the previous problem.
	\item For a matrix $A$, $\vec u,\vec v$ are eigenvectors with different eigenvalues.
		Show that $\vec u$ and $\vec v$ are linearly independent.
	\item Prove that if $A$ is a $4\times 4$ matrix with eigenvalues $2,1,0,-7$, then
		there is a basis for $\R^4$ consisting of eigenvectors of $A$.

\end{enumerate}



%%%%%%%%%%%%%%%%%%%%%%%%%%%%%%%%%%%%%%%%%%%%%%%
% TA instructions
%%%%%%%%%%%%%%%%%%%%%%%%%%%%%%%%%%%%%%%%%%%%%%%
\newpage
{\small
	\begin{center}
		{\bf \mthCourse, \mthTerm}\\ 
		{\bf Tutorial \#\mthTutorialNumber. Instructions for TAs}
	\end{center}
}

\subsection*{Objectives}


\subsection*{Hidden objectives}


\subsection*{Suggestions}

\subsection*{Wrapup}
	Choose a question that most of the class has started but not yet finished,
	or a question that people particularly struggled with.

\subsection*{Solutions}
\begin{enumerate}
	\item
\end{enumerate}
	

\end{document}
