
\begin{exercises}

	\begin{problist}
		\prob
		\begin{enumerate}
			\item Let $\mathcal U:\R^{2}\to\R^{2}$
				be the matrix transformation
				given by $\mat{0&0\\-\sqrt{2}/2
				& -\sqrt{2}/2}$. Further,
				let
				$\mathcal P:\R^{2}\to\R^{2}$
				be the projection onto the
				$y$-axis, and let $\mathcal
				R:\R^{2}\to\R^{2}$ be the rotation
				clockwise by $135^{\circ}$.
				\begin{enumerate}
					\item Find a matrix for
						$\mathcal R
						\circ
						\mathcal P$.

					\item Find a matrix for
						$\mathcal P
						\circ
						\mathcal R$.

					\item Write $\mathcal
						U$ as the composition
						(in some order)
						of $\mathcal
						R$ and $\mathcal
						P$.
				\end{enumerate}

			\item Let $\mathcal V:\R^{2}\to\R^{2}$
				be the matrix transformation
				given by $\mat{0&2\\2&0}$.
				Further, let
				$\mathcal S:\R^{2}\to\R^{2}$
				be the transformation that
				doubles every vector, and
				let
				$\mathcal F:\R^{2}\to\R^{2}$
				be the transformation
				reflecting over the line
				$y=x$.
				\begin{enumerate}
					\item Find a matrix for
						$\mathcal F
						\circ
						\mathcal S$.

					\item Find a matrix for
						$\mathcal S
						\circ
						\mathcal F$.

					\item Write $\mathcal
						V$ as the composition
						(in some order)
						of $\mathcal
						F$ and $\mathcal
						S$.
				\end{enumerate}
		\end{enumerate}

		% Q1 Solutions

		\begin{solution}

			\begin{enumerate}
				\item $(\mathcal R \circ \mathcal
					P)(\mat{1\\0}) =
					\mat{0\\0}$,
					$(\mathcal R \circ \mathcal
					P)(\mat{0\\1}) =
					\mat{\tfrac{1}{\sqrt{2}}\\
					-\tfrac{1}{\sqrt{2}}}$,
					$(\mathcal P \circ \mathcal
					R)(\mat{1\\0}) =
					\mat{0\\ -\tfrac{1}{\sqrt{2}}}$,
					$(\mathcal P \circ \mathcal
					R)(\mat{0\\1}) =
					\mat{0\\ - \tfrac{1}{\sqrt{2}}}$.
					We can use the effect
					of the transformation
					on the standard basis
					to compute the matrix.


					\begin{enumerate}
						\item $M_{R\circ
							P}=
							\mat{\mathcal
							R(\mathcal
							P(\xhat))&
							\mathcal
							R(
							\mathcal
							P(\yhat))}=
							\mat{0&\tfrac{1}{\sqrt{2}}\\0&-\tfrac{1}{\sqrt{2}}}$

						\item $M_{P\circ
							R}=
							\mat{\mathcal
							P(
							\mathcal
							R(\xhat))&
							\mathcal
							P(\mathcal
							R(\yhat))}=
							\mat{0&0\\-\tfrac{1}{\sqrt{2}}&-\tfrac{1}{\sqrt{2}}}$

						\item Since $M_{\mathcal}
							U=M_{\mathcal
							P
							\circ
							\mathcal
							R}$ and
							$\tfrac{1}{\sqrt{2}}=\tfrac{\sqrt{2}}{2}$,
							we
							can
							assert
							that
							$\mathcal
							U=\mathcal
							P
							\circ
							\mathcal
							R$.
					\end{enumerate}

				\item $(\mathcal F \circ \mathcal
					S)(\mat{1\\0}) =
					\mat{0\\2}$,
					$(\mathcal F \circ \mathcal
					S)(\mat{0\\1}) =
					\mat{2\\0}$, $(\mathcal
					S \circ \mathcal F)(\mat{1\\0})
					= \mat{0\\2}$,
					$(\mathcal S \circ \mathcal
					F)(\mat{0\\1}) =
					\mat{2\\0}$. We can use
					the effect of the transformation
					on the standard basis
					to compute the matrix.
					\begin{enumerate}
						\item $M_{\mathcal
							F
							\circ
							\mathcal
							S}=\mat{\mathcal
							F(
							\mathcal
							S(\xhat))&
							\mathcal
							F(\mathcal
							S(\yhat))}=\mat{0&2\\2&0}$

						\item $M_{\mathcal
							S
							\circ
							\mathcal
							F}=\mat{\mathcal
							S(
							\mathcal
							F(\xhat))&
							\mathcal
							S(\mathcal
							F(\yhat))}=\mat{0&2\\2&0}$

						\item $\mathcal
							S\circ
							\mathcal
							F=
							\mathcal
							V =
							\mathcal
							F
							\circ
							\mathcal
							S$.
					\end{enumerate}
			\end{enumerate}
		\end{solution}

		\prob Let $\mathcal A: \R^{2}\to \R^{2}$ and
		$\mathcal B: \R^{2}\to \R^{2}$ be matrix
		transformations with matrices
		\[
			M_{\mathcal A}= \mat{2&2\\1&3}\quad
			\text{and}\quad M_{\mathcal B}= \mat{3&2\\0&4}
		\]
		 and let $M_{\mathcal T}$ be the matrix for
		$\mathcal T=\mathcal A\circ\mathcal B$.
		\begin{enumerate}
			\item Find $M_{\mathcal T}$ by computing
				input-output pairs for $\mathcal
				T$.

			\item Find $M_{\mathcal T}$ by using
				matrix multiplication applied
				to $M_{\mathcal A}$ and $M_{\mathcal
				B}$.
		\end{enumerate}

		% Q2 Solutions

		\begin{solution}

			\begin{enumerate}
				\item $M_{\mathcal T}\mat{1\\0}=
					M_{\mathcal A}M_{\mathcal
					B}\mat{1\\0}= M_{\mathcal
					A}\mat{3&2\\0&4}\mat{1\\0}=
					\mat{2&2\\1&3}\mat{3\\0}=
					\mat{6\\3}$ \\ $M_{\mathcal
					T}\mat{0\\1}= M_{\mathcal
					A}M_{\mathcal B}\mat{0\\1}=
					M_{\mathcal A}\mat{3&2\\0&4}\mat{0\\1}=
					\mat{2&2\\1&3}\mat{2\\4}=
					\mat{12\\14}$.
					Therefore
					$M_{\mathcal T}=
					\mat{6&12\\3&14}$

				\item $M_{\mathcal T}= M_{\mathcal
					A}M_{\mathcal B}=
					\mat{2&2\\1&3}\mat{3&2\\0&4}=
					\mat{6&12\\3&14}$
			\end{enumerate}
		\end{solution}

		\prob Let $\mathcal A: \R^{3}\to \R^{2}$ and
		$\mathcal B: \R^{2}\to \R^{1}$ be matrix transformations
		with matrices
		\[
			M_{\mathcal A}= \mat{2&1&0\\2&3&0}\quad
			and \quad M_{\mathcal B}= \mat{3&2}
		\]
		 and let $M_{\mathcal T}$ be the matrix for $\mathcal
		T=\mathcal B\circ \mathcal A$.
		\begin{enumerate}
			\item Find $M_{\mathcal T}$ by computing
				input-output pairs for $\mathcal
				T$.

			\item Find $M_{\mathcal T}$ by using
				matrix multiplication applied
				to $M_{\mathcal A}$ and $M_{\mathcal
				B}$.
		\end{enumerate}

		% Q3 Solutions

		\begin{solution}

			\begin{enumerate}
				\item $\mathcal T = \mathcal
					B \circ \mathcal A$,
					thus
					$\mathcal T : \R^{3}
					\to \R$. \\ $\mathcal
					T \mat{1\\0\\0}= (\mathcal
					B \circ \mathcal A)
					\mat{1\\0\\0}=
					\mathcal B \mat{2\\2}=
					10$

					$\mathcal T \mat{0\\1\\0}=
					(\mathcal B \circ \mathcal
					A) \mat{0\\1\\0}=
					\mathcal B \mat{1\\3}=
					9$

					$\mathcal T \mat{0\\0\\1}=
					(\mathcal B \circ \mathcal
					A) \mat{0\\0\\1}=
					\mathcal B \mat{0\\0}=
					0$. Therefore
					$M_{\mathcal T}=
					\mat{10&9&0}$

				\item $M_{\mathcal T}= M_{\mathcal
					B}M_{\mathcal A}=
					\mat{3&2}\mat{2&1&0\\2&3&0}=
					\mat{10&9&0}$
			\end{enumerate}
		\end{solution}
	\end{problist}
\end{exercises}