\begin{exercises}
	\begin{problist}
		\prob 
		\begin{enumerate}
			\item Let $\mathcal U:\R^{2}\to\R^{2}$ be the matrix transformation
				given by $\mat{0&0\\-\sqrt{2}/2 & -\sqrt{2}/2}$.
				Further, let $\mathcal P:\R^{2}\to\R^{2}$ be the
				projection onto the $y$-axis, and let $\mathcal R:\R^{2}\to\R^{2}$
				be the rotation clockwise by $135^{\circ}$.
				\begin{enumerate}
					\item Find a matrix for $\mathcal R \circ \mathcal P$.

					\item Find a matrix for $\mathcal P \circ \mathcal R$.

					\item Write $\mathcal U$ as the composition
						(in some order) of $\mathcal R$ and
						$\mathcal P$.
				\end{enumerate}

			\item Let $\mathcal V:\R^{2}\to\R^{2}$ be the matrix transformation
				given by $\mat{0&2\\2&0}$.
				Further, let $\mathcal S:\R^{2}\to\R^{2}$ be the
				transformation that doubles every vector, and let
				$\mathcal F:\R^{2}\to\R^{2}$ be the
				transformation reflecting over the line $y=x$.
				\begin{enumerate}
					\item Find a matrix for $\mathcal F \circ \mathcal S$.

					\item Find a matrix for $\mathcal S \circ \mathcal F$.

					\item Write $\mathcal U$ as the composition
						(in some order) of $\mathcal F$ and
						$\mathcal S$.
				\end{enumerate}
		\end{enumerate}
		
		% Q1 Solutions
		\begin{solution}
		    \begin{enumerate}
			\item 
				Note that $P$ and $R$ are both linear transformations. So we can find their corresponding matrices, $M_{\mathcal P}$ and $M_{\mathcal R}$. 
				$M_{\mathcal P}=\mat{0&0\\0&1}$ since $\mathcal P\mat{1\\0}=\mat{0\\0}$ and $\mathcal P\mat{0\\1}=\mat{0\\1}$.
				For the first column of $M_{\mathcal R}$ we calculate $\mathcal R\mat{1\\0}=\mat{-\cos{45^{\circ}}\\-\sin{45^{\circ}}}=\mat{-\tfrac{1}{\sqrt{2}}\\-\tfrac{1}{\sqrt{2}}}$
				and for the second we compute $\mathcal R\mat{0\\1}=\mat{\cos{45^{\circ}}\\-\sin{45^{\circ}}}=\mat{\tfrac{1}{\sqrt{2}}\\-\tfrac{1}{\sqrt{2}}}$.
				Thus we get $M_{\mathcal R}=\mat{-\tfrac{1}{\sqrt{2}}&\tfrac{1}{\sqrt{2}}\\-\tfrac{1}{\sqrt{2}}&-\tfrac{1}{\sqrt{2}}}$

				\begin{enumerate}
					\item We want to compute $M_{R\circ P}$ and we get that $M_{R\circ P}= M_\mathcal R M_{\mathcal P}=\mat{-\tfrac{1}{\sqrt{2}}&\tfrac{1}{\sqrt{2}}\\-\tfrac{1}{\sqrt{2}}&-\tfrac{1}{\sqrt{2}}}\mat{0&0\\0&1}=\mat{0&\tfrac{1}{\sqrt{2}}\\0&-\tfrac{1}{\sqrt{2}}}$

					\item We want to compute $M_{P\circ R}$ and we get that $M_{P\circ R}= M_\mathcal P M_{\mathcal R}=\mat{0&0\\0&1}\mat{-\tfrac{1}{\sqrt{2}}&\tfrac{1}{\sqrt{2}}\\-\tfrac{1}{\sqrt{2}}&-\tfrac{1}{\sqrt{2}}}=\mat{0&0\\-\tfrac{1}{\sqrt{2}}&-\tfrac{1}{\sqrt{2}}}$

					\item Since $M_\mathcal U=M_{\mathcal P \circ \mathcal R}$ and $\tfrac{1}{\sqrt{2}}=\tfrac{\sqrt{2}}{2}$, we can assert that $\mathcal U=\mathcal P \circ \mathcal R$.
				\end{enumerate}
				\item Note $M_\mathcal V=\mat{0&2\\2&0}$. Notice that $\mathcal S$ is a linear transformation. (On a test you might need to prove this before writing a matrix.) We observe what $\mathcal S$ does to the standard basis vectors: $\mathcal S\mat{1\\0}=\mat{2\\0}$ and $\mathcal S\mat{0\\1}=\mat{0\\2}$. Hence, $M_\mathcal S=\mat{2&0\\0&2}$. Note that $\mathcal F$ is given by $\mathcal F\mat{x\\y}=\mat{y\\x}$ which is also a linear transformation and so we can find its matrix: $M_\mathcal F=\mat{0&1\\1&0}$.
				\begin{enumerate}
				    \item $M_{\mathcal F \circ \mathcal S}=M_\mathcal F M_\mathcal S=\mat{0&1\\1&0}\mat{2&0\\0&2}=\mat{0&2\\2&0}$
				    \item $M_{\mathcal S \circ \mathcal F}=M_\mathcal S M_\mathcal F=\mat{2&0\\0&2}\mat{0&1\\1&0}=\mat{0&2\\2&0}$
				    \item $\mathcal S\circ \mathcal F= \mathcal V = \mathcal F \circ \mathcal S$.
				\end{enumerate}
			\end{enumerate}	
		\end{solution}

		\prob Let $\mathcal A: \R^{2} \to \R^{2}$ and
		$\mathcal B: \R^{2} \to \R^{2}$ be matrix transformations with
		matrices
		\[
			M_{\mathcal A}= \mat{2&2\\1&3}\quad \text{and} \quad M_{\mathcal
			B}= \mat{3&2\\0&4}
		\]
		 and let $M_{\mathcal T}$ be the matrix for $\mathcal T=\mathcal A\circ\mathcal B$.
		\begin{enumerate}
			\item Find $M_{\mathcal T}$ by computing input-output pairs for $\mathcal
				T$.

			\item Find $M_{\mathcal T}$ by using matrix multiplication applied
				to $M_{\mathcal A}$ and $M_{\mathcal B}$.
		\end{enumerate}
		
		% Q2 Solutions
	    \begin{solution}
	        \begin{enumerate}
	            \item $M_{\mathcal T} \mat{1\\0} 
	            = M_{\mathcal A} M_{\mathcal B} \mat{1\\0}
	            = M_{\mathcal A} \mat{3&2\\0&4} \mat{1\\0}
	            = \mat{2&2\\1&3} \mat{3\\0} = \mat{6\\3}$
	            \\
	            $M_{\mathcal T} \mat{0\\1} 
	            = M_{\mathcal A} M_{\mathcal B} \mat{0\\1}
	            = M_{\mathcal A} \mat{3&2\\0&4} \mat{0\\1}
	            = \mat{2&2\\1&3} \mat{2\\4} = \mat{12\\14}$.
	            Therefore $M_{\mathcal T} = \mat{6&12\\3&14}$
	            
	            \item $M_{\mathcal T}
	            = M_{\mathcal A} M_{\mathcal B}
	            =  \mat{2&2\\1&3} \mat{3&2\\0&4}
	            = \mat{6&12\\3&14}$
	        \end{enumerate}
	    \end{solution}

		\prob Let $\mathcal A: \R^{3} \to \R^{2}$ and
		$\mathcal B: \R^{2} \to \R^{1}$ be matrix transformations with
		matrices
		\[
			M_{\mathcal A}= \mat{2&1&0\\2&3&0}\quad and \quad M_{\mathcal
			B}= \mat{3&2}
		\]
		 and let $M_{\mathcal T}$ be the matrix for $\mathcal T=\mathcal B\circ \mathcal A$.
		\begin{enumerate}
			\item Find $M_{\mathcal T}$ by computing input-output pairs for $\mathcal
				T$.

			\item Find $M_{\mathcal T}$ by using matrix multiplication applied
				to $M_{\mathcal A}$ and $M_{\mathcal B}$.
		\end{enumerate}
		
		% Q3 Solutions
% 		\begin{solution}
            \begin{enumerate}
                \item $\mathcal T = \mathcal B \circ \mathcal A$, thus $\mathcal T : \R^3 \to \R$. \\
                $\mathcal T \mat{1\\0\\0} 
                = (\mathcal B \circ \mathcal A) \mat{1\\0\\0}
                = \mathcal B \mat{2\\2} = 10$
                
                $\mathcal T \mat{0\\1\\0} 
                = (\mathcal B \circ \mathcal A) \mat{0\\1\\0}
                = \mathcal B \mat{1\\3} = 9$
                
                $\mathcal T \mat{0\\0\\1} 
                = (\mathcal B \circ \mathcal A) \mat{0\\0\\1}
                = \mathcal B \mat{0\\0} = 0$. Therefore $M_{\mathcal T} = \mat{10&9&0}$
                
                \item $M_{\mathcal T} = M_{\mathcal B} M_{\mathcal A}
                = \mat{3&2} \mat{2&1&0\\2&3&0} = \mat{10&9&0}$
            \end{enumerate}
% 		\end{solution}
	\end{problist}
\end{exercises}
