Consider the following situation: you're designing a 3d video game, but your users only have 2d screens.
Or, you have a 900-dimensional dataset, but you want to visualize it on a continuum (i.e., as a line). 
Each of these is an example
of finding the best approximation to particular points given restrictions. In general, this operation
is called a \emph{projection}\footnote{ What we define as a \emph{projection} is sometimes
called the \emph{orthogonal projection} to distinguish it from other types of projections.}, 
and in the world of linear algebra, it has a very particular meaning.

\SavedDefinitionRender{Projection}

Let $\mathcal P_{xy}\subseteq\R^3$ be the $xy$-plane in $\R^3$ and let $\vec v=\mat{1\\2\\3}$. Intuitively, 
$\Proj_{\mathcal P_{xy}}\!\!\vec v$ is the ``shadow'' that  $\vec v$ would casts on ${\mathcal P_{xy}}$ if 
the sun were directly overhead.
Upon drawing a picture, we conclude $
	\Proj_{\mathcal P_{xy}}\vec v=\mat{1\\2\\0}.
$

\begin{center}
  \begin{tikzpicture}
    \begin{axis}[grid=major,view={60}{40},%z buffer=sort,
	    scale=1.5,
	    axis lines=middle,
	    enlargelimits={abs=0.5},
	    zmin=0, zmax=4, xmin=-2, xmax=3, ymin=-2, ymax=3,
	    xtick={-4,...,4},
	    ytick={-4,...,4},
	    ztick={-4,...,4},
	    xlabel=$x$,
	    ylabel=$y$,
	    zlabel=$z$,
	    %xticklabels={,,}, yticklabels={,,}, zticklabels={,,}
	    ]
		%\addplot3 [myorange, data cs=cart,surf,domain=-3:3,samples=2, opacity=0.5] {0};
		\coordinate (O) at (axis cs:0,0,0);
		\coordinate (V) at (axis cs:1,2,3);
		\coordinate (P) at (axis cs:1,2,0);
		\draw[dashed, black!50] (P) -- (V);
		\draw[->, thick, mypink] (O) -- (V) node[above right] {$\vec v$};
		\draw[->, thick, myorange] (O) -- (P) node[right] {$\Proj_{\mathcal P_{xy}}\vec v$};
	%\addplot3[domain=4:30,samples=80,samples y=0,mark=none,orange, opacity=0.5,ultra thick]
	%    ({x},{118.89/x},{2*x});
    \end{axis}
  \end{tikzpicture}
\end{center}

Continuing, let $\ell_y\subseteq \R^3$ be the $y$-axis in $\R^3$. It's a little bit harder to visualize what $\Proj_{\ell_y}\vec v$
is, so let's appeal to some definitions.

By definition, every vector in $\ell_y$ takes the form $\vec u_t=\mat{0\\t\\0}$ for some $t\in \R$. The distance
between $\vec u_t$ and $\vec v$ is
\[
	\Norm{\vec u_t-\vec v}=\norm*{\mat{0\\t\\0}-\mat{1\\2\\3}} = \sqrt{1^2+(t-2)^2+3^2}.
\]
Since $(t-2)^2$ is always positive, the quantity $\sqrt{1^2+(t-2)^2+3^2}$ is minimized when $(t-2)^2=0$; that is,
when $t=2$. Thus, we see $\vec u_2$ is the closest vector in $\ell_y$ to $\vec v$ and so,
\[
	\Proj_{\ell_y}\vec v=\vec u_2=\mat{0\\2\\0}.
\]


\begin{example}
	Let $\ell\subseteq \R^2$ be the line given in vector form by $\vec x=t\mat{1\\1}+\mat{3\\-2}$,
	and let $\vec v=\mat{-1\\-1}$. Use the definition of projection to find $\Proj_\ell \vec v$.

	Let $\vec u_t=t\mat{1\\1}+\mat{3\\-2}\in\ell$.
	By definition, the distance between $\vec v$ and $\vec u_t$ is given by
	\[
		\Norm{\vec u_t-\vec v}=\norm*{\left(t\mat{1\\1}+\mat{3\\-2}\right)-\mat{-1\\-1}} = \norm*{\mat{t+2\\t-3}}
	    =\sqrt{2t^2+6t+17}.
	\]
	The quantity $2t^2+6t+17$ is minimized when $t=-\frac{3}{2}$, and so the closest point
	in $\ell$ to $\vec v$ is $\vec u_{-3/2}$. Thus,
	\[
	    \Proj_\ell \vec v=-\tfrac{3}{2}\mat{1\\1}+\mat{3\\-2}=\mat{3/2\\-7/2}.
	\]
\end{example}

Every example of a projection so far shares a geometric property. In the case of lines and planes,
the vector from the  projection to the original point is a normal vector for the line or plane (provided it's non-zero).

\begin{center}
	\begin{tikzpicture}
		\def\dotMarkRightAngle[size=#1](#2,#3,#4){%
		 \draw ($(#3)!#1!(#2)$) -- 
		       ($($(#3)!#1!(#2)$)!#1!90:(#2)$) --
		       ($(#3)!#1!(#4)$);
		 \path (#3) -- ($($(#3)!#1!(#2)$)!#1!90:(#2)$);
		}
		\begin{axis}[
		    anchor=origin,
		    name=plot1,
		    disabledatascaling,
		    xmin=-2,xmax=4,
		    ymin=-1,ymax=3,
			xtick={-4,...,4},
			ytick={-2,...,4},
		    x=1cm,y=1cm,
		    grid=both,
		    grid style={line width=.1pt, draw=gray!10},
		    %major grid style={line width=.2pt,draw=gray!50},
		    axis lines=middle,
		    minor tick num=0,
		    enlargelimits={abs=0.5},
		    axis line style={latex-latex},
		    ticklabel style={font=\tiny,fill=white},
		    xlabel style={at={(ticklabel* cs:1)},anchor=north west},
		    ylabel style={at={(ticklabel* cs:1)},anchor=south west}
		]
			\begin{scope}[cm={1,0,0,1,(1,1)}]
				\coordinate (A) at (1,2);
				\coordinate (O) at (0,0);
				\coordinate (B) at (2,-1);
				\dotMarkRightAngle[size=6pt](B,O,A);
				\draw[mypink, thick] ($-2*(-2,1)$) -- ($2*(-2,1)$);
				\draw[myorange, thick, dashed, ->] (0,0) -- (1,2) node[midway, below right] {$\vec v-\proj_X\vec v$};
				\fill[Violet]  (1,2) circle (2pt) node[above right] {$\vec v$};
				\fill[mygreen] (0,0) node[below left, xshift=6pt, yshift=-2pt] {$\Proj_X\vec v$} circle (2pt);
			\end{scope}
		\end{axis}
		\node[mypink, above right] at (2.8,.1) {$X$};
	\end{tikzpicture}
\end{center}

Stated precisely, if $X$ is a line or plane and $\vec v\notin X$ is a vector, then $\vec v-\Proj_X\vec v$ is a
normal vector for $X$. Using this fact, we can find projections onto lines and planes without needing to compute
any distances!

\begin{example}
	Let $\ell\subseteq \R^2$ be the line given in vector form by $\vec x=t\mat{1\\1}+\mat{3\\-2}$,
	and let $\vec v=\mat{-1\\-1}$. Use the fact that $\vec v-\Proj_\ell\vec v$ is a normal vector to $\ell$
	to find Find $\Proj_\ell \vec v$.

	Since $\vec v-\Proj_\ell\vec v$ is a normal vector to $\ell$, we know $\vec v-\Proj_\ell\vec v$ is orthogonal to $\vec d=\mat{1\\1}$. 
	Let $\mat{x\\y}=\Proj_\ell\vec v$ for some unknown $x,y\in\R$. We now know
	\[
		(\vec v-\Proj_\ell\vec v)\cdot\vec d=\left(\mat{-1\\-1}-\mat{x\\y}\right)\cdot\mat{1\\1}
		=\mat{-1-x\\-1-y}\cdot \mat{1\\1}=-2-x-y=0.
	\]
	That is, 
	\begin{equation}
	\label{EQDOTZERO}
	    x+y=-2.
	\end{equation}
	Also, since $\Proj_\ell\vec v\in\ell$, we know $\Proj_\ell\vec v$ know
	\[
    		\mat{x\\y}=t\mat{1\\1}+\mat{3\\-2}=\mat{t+3\\t-2}.
	\]
	From this, we have that $x-t=3$ and $y-t=-3$. Combined with Equation \eqref{EQDOTZERO}, we have
	three equations and three unknowns which produce the following system of linear equations.
	\[
		\systeme{x+y=-2,x-t=3,y-t=-2}
	\]
	Solving this system, we conclude that $x=3/2$ and $y=-7/2$ (we don't care about the value of $t$). Therefore
	 $\Proj_\ell\vec v = \mat{3/2\\-7/2}$. 
\end{example}

\begin{emphbox}[Takeaway]
	When projecting onto lines an planes, right angles appear in key places.
\end{emphbox}

\Heading{Projections Onto Other Sets}
For projections onto lines and planes, we can use what we know about normal vectors to simplify our life.
The same is true when projecting onto other sets, but we must always keep the definition in mind.

\begin{example}
	Let $\mathcal T\subseteq \R^2$ be the filled in triangle with vertices $\mat{0\\0}$, 
	$\mat{1\\0}$, and $\mat{0\\2}$, and let
	\[
		\vec a=\mat{1/4\\1/4}\qquad \vec b=\mat{1\\1}\qquad \vec c=\matc{3\\1/2}.
	\]
	Find $\Proj_{\mathcal T}\vec a$, $\Proj_{\mathcal T}\vec b$, and  $\Proj_{\mathcal T}\vec c$.

	We'll start by drawing a picture.

\begin{center}
	\begin{tikzpicture}
		\begin{axis}[
		    anchor=origin,
		    name=plot1,
		    disabledatascaling,
		    xmin=-1,xmax=3,
		    ymin=-1,ymax=2,
			xtick={-4,...,4},
			ytick={-2,...,4},
		    x=1cm,y=1cm,
		    grid=both,
		    grid style={line width=.1pt, draw=gray!10},
		    %major grid style={line width=.2pt,draw=gray!50},
		    axis lines=middle,
		    minor tick num=0,
		    enlargelimits={abs=0.5},
		    axis line style={latex-latex},
		    ticklabel style={font=\tiny,fill=white},
		    xlabel style={at={(ticklabel* cs:1)},anchor=north west},
		    ylabel style={at={(ticklabel* cs:1)},anchor=south west}
		]
			\coordinate (A) at (1/4,1/4);
			\coordinate (B) at (3,1/2);
			\coordinate (C) at (1,1);
			\fill[blue, opacity=.3] (0,0) -- (1,0) -- (0,2) node[above right, opacity=.7] {$\mathcal T$};
			\fill[mypink] (A) circle (2pt) node[above right] {$\vec a$};
			\fill[mygreen] (B) circle (2pt) node[above right] {$\vec c$};
			\fill[myorange] (C) circle (2pt) node[above right] {$\vec b$};
		\end{axis}
	\end{tikzpicture}
\end{center}

	From the picture, we see that $\vec a\in \mathcal T$ and so
	\[
		\Proj_{\mathcal T}\vec a=\vec a.
	\]
	We also see that $\vec b$ is closest to the hypotenuse of $\mathcal T$, and so $\Proj_{\mathcal T}\vec b$ is
	the same as the projection of $\vec b$ onto the line $y=-2x+2$. Computing, we find
	\[
		\Proj_{\mathcal T}\vec b=\mat{3/5\\4/5}.
	\]
	Finally, drawing concentric circles centered at $\vec c$, we see that the lower-right corner of $\mathcal T$
	is the closest point in $\mathcal T$ to $\vec c$.

\begin{center}
	\begin{tikzpicture}
		\begin{axis}[
		    anchor=origin,
		    name=plot1,
		    disabledatascaling,
		    xmin=-1,xmax=3,
		    ymin=-1,ymax=2,
			xtick={-4,...,4},
			ytick={-2,...,4},
		    x=1cm,y=1cm,
		    grid=both,
		    grid style={line width=.1pt, draw=gray!10},
		    %major grid style={line width=.2pt,draw=gray!50},
		    axis lines=middle,
		    minor tick num=0,
		    enlargelimits={abs=0.5},
		    axis line style={latex-latex},
		    ticklabel style={font=\tiny,fill=white},
		    xlabel style={at={(ticklabel* cs:1)},anchor=north west},
		    ylabel style={at={(ticklabel* cs:1)},anchor=south west}
		]
			\coordinate (a) at (1/4,1/4);
			\coordinate (b) at (3,1/2);
			\coordinate (c) at (1,1);
			\fill[blue, opacity=.3] (0,0) -- (1,0) -- (0,2) node[above right, opacity=.7] {$\mathcal T$};
			\fill[mygreen] (b) circle (2pt) node[above right] {$\vec c$};
			\draw[dashed] (b) circle (2.061) circle (1.6) circle (2.5);
		\end{axis}
	\end{tikzpicture}
\end{center}

	And so, 
	\[
		\Proj_{\mathcal T}\vec c =\mat{1\\0}.
	\]
\end{example}

\Heading{Subtleties of Projections}
You might be wondering, what is $\Proj_X\vec v$ if $\vec v$ is equidistant from \emph{two}
closest points in $X$? Or, what if $X$ is an \emph{open} set (for example, an open interval in $\R^1$)?
Then there might not be a closest point in $X$ to $\vec v$. In both these cases, we say $\Proj_X\vec v$
is \emph{undefined}.

Formally, for a fixed set $X$, we consider $P(\vec v)=\Proj_X\vec v$ as a function that inputs and outputs
vectors. And, as a function, $P$ has a domain consisting of exactly the vectors $\vec v$ for which $P(\vec v)$
is defined. As it happens, if $X$ is a line or a plane in $\R^n$, the domain of $P$ is all of $\R^n$, and in this text,
we will be sensible and only ask about projections that exist. 

\Heading{Vector Components}

We've seen before that dot products can be used to measure how much one
vector points in the direction of another. But, we can go further. Suppose 
$\vec v\neq \vec 0$ and $\vec u$ are vectors. We might want to \emph{decompose}
$\vec u$ into the sum of two vectors, one which is in the direction of $\vec v$
and the other which is orthogonal to $\vec v$. The tool to that does this is the \emph{vector component}.

\SavedDefinitionRender{Component}

From the definition, it's obvious that
\[
	\vec u=\Vcomp_{\vec v}\vec u + (\vec u-\Vcomp_{\vec v}\vec u)
\]
is a decomposition of $\vec u$ into the sum of two vectors, one ($\Vcomp_{\vec v}\vec u$) is in the direction of $\vec v$,
and the other ($\vec u-\Vcomp_{\vec v}\vec u$) is orthogonal to $\vec v$.

\begin{example}
	Find the vector component of $\vec a=\mat{1\\2}$ in the direction of $\vec b=\mat{1\\1}$.

	Since $\Vcomp_{\vec b}\vec a$ is a vector in the direction of $\vec b$, we know
	\[
		\Vcomp_{\vec b}\vec a = k\vec b
	\]
	for some $k\in \R$. Since $\vec a-\Vcomp_{\vec b}\vec a$ is orthogonal to $\vec b$, we know
	\[
		(\vec a-\Vcomp_{\vec b}\vec a)\cdot \vec b=0.
	\]
	Combining these facts, we see
	\[
		(\vec a-\Vcomp_{\vec b}\vec a)\cdot \vec b = \bigg(\underbrace{\mat{1\\2}}_{\vec a}-\underbrace{\mat{k\\k}}_{k\vec b}\bigg)\cdot 
		\underbrace{\mat{1\\1}}_{\vec b} =(1-k)+(2-k)=3-2k=0,
	\]
	and so $k=3/2$. Therefore
	\[
		\Vcomp_{\vec b}\vec a=k\vec b=\mat{3/2\\3/2}.
	\]
\end{example}

Since we'll be computing vector components often, let's try to find a formula for $\Vcomp_{\vec v}\vec u$.

By definition, $\Vcomp_{\vec v}\vec u$ is a vector in the direction of $\vec v$, so
\[
	\Vcomp_{\vec v}\vec u = k\vec v.
\]
Further, from the definition $\vec u-\Vcomp_{\vec v}\vec u$ is orthogonal to $\vec v$, and so
\[
	\vec v\cdot (\vec u-\Vcomp_{\vec v}\vec u) = \vec v\cdot (\vec u-k\vec v)=\vec v\cdot \vec u-k\vec v\cdot \vec v=0,
\]
Because $\vec v\neq \vec 0$, we know $\vec v\cdot \vec v\neq 0$. Therefore, we may rearrange and solve for $k$ to find
\[
	k=\frac{\vec v\cdot \vec u}{\vec v\cdot \vec v},
\]
which means
\[
	\Vcomp_{\vec v}\vec u = \left(\frac{\vec v\cdot \vec u}{\vec v\cdot \vec v}\right)\vec v.
\]

\Heading{The Relationship Between Vector Components and Projections}

Vector components and projections onto lines are closely related. So closely related that many textbooks
use the single word \emph{projection} to talk about both vector components and projections\footnote{ We will not.}. Let's take a moment
to explore this relationship.

Let $\vec v=\mat{2\\4}$ and $\vec u=\mat{3\\1}$ and let $\ell=\Span\Set{\vec v}$. Drawing a picture of
$\ell$, $\vec u$, and $\Proj_\ell\vec u$, we see that $\Proj_\ell\vec u$ satisfies all the properties
of $\Vcomp_{\vec v}\vec u$.

\begin{center}
	\begin{tikzpicture}
		\def\dotMarkRightAngle[size=#1](#2,#3,#4){%
		 \draw ($(#3)!#1!(#2)$) -- 
		       ($($(#3)!#1!(#2)$)!#1!90:(#2)$) --
		       ($(#3)!#1!(#4)$);
		 \path (#3) -- ($($(#3)!#1!(#2)$)!#1!90:(#2)$);
		}
		\begin{axis}[
		    anchor=origin,
		    name=plot1,
		    disabledatascaling,
		    xmin=-1,xmax=4,
		    ymin=-1,ymax=6,
			xtick={-4,...,8},
			ytick={-2,...,8},
		    x=1cm,y=1cm,
		    grid=both,
		    grid style={line width=.1pt, draw=gray!10},
		    %major grid style={line width=.2pt,draw=gray!50},
		    axis lines=middle,
		    minor tick num=0,
		    enlargelimits={abs=0.5},
		    axis line style={latex-latex},
		    ticklabel style={font=\tiny,fill=white},
		    xlabel style={at={(ticklabel* cs:1)},anchor=north west},
		    ylabel style={at={(ticklabel* cs:1)},anchor=south west}
		]
			\coordinate (A) at (2,4);
			\coordinate (O) at (0,0);
			\coordinate (P) at (1,2);
			\coordinate (B) at (3,1);
			\dotMarkRightAngle[size=6pt](O,P,B);
			\draw[black!50,  dashed] (P) -- (B);
			%\dotMarkRightAngle[size=6pt](O,P,B);
			%\draw[black!80,  dashed] (P) -- (B);
			\draw[myorange, thick] ($-2*(A)$) -- ($2*(A)$);
			%\draw[black!80, very thick, dashed, ->] (0,0) -- (A) node[right] {$\vec v$};
			\fill[mygreen]  (B) circle (2pt) node[above right] {$\vec u$};
			\fill[mypink]  (P) circle (2pt) node[above left, yshift=4pt, xshift=4pt] {$\Proj_\ell\vec u$};
			%\draw[mypink,  ->, xshift=-2pt, yshift=2pt] (0,0) -- (1,2);
		\end{axis}
		\node[myorange, above left] at (2.5,4.75) {$\ell=\Span\Set{\vec v}$};
	\end{tikzpicture}
	~~~~~
	\begin{tikzpicture}
		\def\dotMarkRightAngle[size=#1](#2,#3,#4){%
		 \draw ($(#3)!#1!(#2)$) -- 
		       ($($(#3)!#1!(#2)$)!#1!90:(#2)$) --
		       ($(#3)!#1!(#4)$);
		 \path (#3) -- ($($(#3)!#1!(#2)$)!#1!90:(#2)$);
		}
		\begin{axis}[
		    anchor=origin,
		    name=plot1,
		    disabledatascaling,
		    xmin=-1,xmax=4,
		    ymin=-1,ymax=6,
			xtick={-4,...,8},
			ytick={-2,...,8},
		    x=1cm,y=1cm,
		    grid=both,
		    grid style={line width=.1pt, draw=gray!10},
		    %major grid style={line width=.2pt,draw=gray!50},
		    axis lines=middle,
		    minor tick num=0,
		    enlargelimits={abs=0.5},
		    axis line style={latex-latex},
		    ticklabel style={font=\tiny,fill=white},
		    xlabel style={at={(ticklabel* cs:1)},anchor=north west},
		    ylabel style={at={(ticklabel* cs:1)},anchor=south west}
		]
			\coordinate (A) at (2,4);
			\coordinate (O) at (0,0);
			\coordinate (P) at (1,2);
			\coordinate (B) at (3,1);
			\dotMarkRightAngle[size=6pt](O,P,B);
			\draw[black!50,  dashed] (P) -- (B);
			%\draw[myorange, thick] ($-2*(A)$) -- ($2*(A)$);
			\draw[mygreen, ->, thick] (O) -- (B) node[above right] {$\vec u$};
			\draw[mypink,  ->, thick, xshift=-2pt, yshift=1pt] (0,0) -- (1,2) node[above left, yshift=4pt, xshift=4pt] {$\Comp_{\vec v}\vec u$};
			\draw[black!80, very thick, dashed, ->, xshift=0pt] (0,0) -- (2,4) node[right] {$\vec v$};
		\end{axis}
		%\node[myorange, above left] at (2.5,4.75) {$\ell=\Span\Set{\vec v}$};
	\end{tikzpicture}
\end{center}

Since $\ell=\Span\Set{\vec v}$ and $\Proj_\ell\vec u\in\ell$, we know that $\Proj_\ell\vec u$ is in the direction
of $\vec v$. Further, using geometric arguments, we know $\vec u-\Proj_\ell\vec u$ is a normal vector for $\ell$
and is therefore orthogonal to its direction vector $\vec v$! What's more, we didn't
use anything in particular about $\vec u$ and $\vec v$ when making this argument (other than $\vec v\neq \vec 0$). This means,
we have established a general fact. 

\begin{theorem}
	For vectors $\vec u$ and $\vec v\neq 0$, we have
	\[
		\Proj_{\Span\Set{\vec v}}\vec u=\Vcomp_{\vec v}\vec u.
	\]
\end{theorem}

This is great news because vector components are easy to compute using dot products while projections are usually hard to compute.

\begin{example}
	Compute the projection of $\vec a=\mat{3\\7}$ onto $\ell=\Span\Set*{\mat{1\\-4}}$.

	Let $\vec b = \mat{1\\-4}$. 
	Since $\ell=\Span\Set{\vec b}$ and $\vec b\neq \vec 0$, by the theorem above, we have
	\[
	    \Proj_\ell \vec a = \Vcomp_{\vec b} \vec a = \left(\frac{\vec d\cdot \vec a}{\vec b\cdot \vec b}\right)\vec b 
	    = -\tfrac{3-28}{1+16}\mat{1\\-4}=\mat{-25/17\\100/17}.
	\]
\end{example}

It's worth noting, however, that vector components are equal to projections \emph{only in the case when you're
projecting onto a span}. In general, projections and vector components are unrelated. 

\begin{example}
	Let $\vec a=\mat{3\\7}$ , $\vec b=\mat{1\\-4}$, and let $\ell$ be the line given in vector form by
	$\vec x=t\vec b+\vec a$. Show that $\Proj_\ell\vec a\neq \Vcomp_{\vec b}\vec a$.

	By definition, $\Proj_{\ell}\vec a$ is the closest point in $\ell$ to $\vec a$. Since $\vec a\in\ell$, we must
	have 	
	\[
	    \Proj_{\ell}\vec a = \vec a = \mat{3\\7}.
	\]
	We already computed $\Vcomp_{\vec b}\vec a=\mat{-25/17\\100/17}$ in the previous example, and so we see
	\[
		\Vcomp_{\vec b}\vec a=\mat{-25/17\\100/17}\neq \mat{3\\7}=\Proj_{\ell}\vec b.
	\]
\end{example}

\begin{emphbox}[Takeaway]
	When projecting onto the span of a single vector, you can use vector components as a computational shortcut, 
	but if the set isn't a span, you cannot.
\end{emphbox}

