Matrix-vector multiplication gives a compact way to represent systems of linear equations.

Consider the system
\begin{equation}
	\label{EQREGSYS}
	\sysdelim\{.
		\systeme{
			x+2y-2z=-15,
			2x+y-5z=-21,
			x-4y+z=18
		},
\end{equation}
which is equivalent to the vector equation
\[
	\mat{x+2y-2z\\
           2x+y-5z\\
	     x-4y+z}=\mat{-15\\-21\\18}.
\]
But, we can also rewrite \eqref{EQREGSYS} using matrix-vector multiplication:
\[
	\underbrace{\mat{1&2&-2\\2&1&-5\\1&-4&1}}_{A}\mat{x\\y\\z}=\mat{-15\\-21\\18}.
\]
The matrix $A$ on the left is called the \emph{coefficient matrix}\index{coefficient matrix} because it
is made up of the coefficients from equation \eqref{EQREGSYS}.

By using coefficient matrices, every system of linear equations can be rewritten as a single matrix equation
of the form
\[
	A\vec x=\vec b
\]
where $A$ is a coefficient matrix, $\vec x$ is a column vector of variables, and $\vec b$ 
is a column vector of constants.

\begin{example}
	Consider the one equation system
	\begin{equation}
		\label{EQSYS1}
		\systeme{x-4y+z=5}
	\end{equation}
	and the two-equation system
	\begin{equation}
		\label{EQSYS2}
		\systeme{x-4y+z=5, y-z=9}.
	\end{equation}
	Rewrite each system as a single matrix equation.

	For the one equation system, it is equivalent to
	\[
	    \mat{x-4y+z}=\mat{5}.
	\]
	Thus, we can rewrite \eqref{EQSYS1} as:
	\[
	    \mat{1&-4&1}\mat{x\\y\\z}=\mat{5}.
	\]
	For the two-equation system, it is equivalent to
	\[
	     \mat{x-4y+z\\0x+y-z}=\mat{5\\9}.
	\]
	Thus, we can rewrite \eqref{EQSYS2} as:
	\[
	    \mat{1&-4&1\\0&1&-1}\mat{x\\y\\z}=\mat{5\\9}.
	\]
\end{example}

\Heading{Interpretations of Matrix Equations}

The solution set to a system of linear equations, like
\begin{equation}
	\label{EQREGSYSb}
	\sysdelim\{.
		\systeme{
			x+2y-2z=-15,
			2x+y-5z=-21,
			x-4y+z=18
		}
\end{equation}
can be interpreted as the intersection of three planes (or hyperplanes if there were more variables)---each individual equation
specifies one plane and so solutions to the system are in the intersection of the planes specified by each equation.

However, when this system is rewritten in matrix form, the two ways to interpret matrix-vector
multiplication give rise to two additional ways to interpret the solution set.

\Heading{The \emph{Column} Picture}

Using the column interpretation of matrix-vector multiplication, we see that system \eqref{EQREGSYSb} is equivalent to
\[
	\mat{1&2&-2\\2&1&-5\\1&-4&1}\mat{x\\y\\z}=
	     x\mat{1\\2\\1}+y\mat{2\\1\\-4}+z\mat{-2\\-5\\1}
	     =\mat{-15\\-21\\18}.
\]
This means the question, ``What are the solutions to system \eqref{EQREGSYSb}?'' is equivalent to the question, ``What coefficients
allow $\mat{1\\2\\1}$, $\mat{2\\1\\-4}$, and $\mat{-2\\-5\\1}$ to form $\mat{-15\\-21\\18}$ as a linear combination?'' Here,
 $\mat{1\\2\\1}$, $\mat{2\\1\\-4}$, and $\mat{-2\\-5\\1}$ are the columns of the coefficient matrix.


\Heading{The \emph{Row} Picture}
The row interpretation gives us another perspective. Let $\vec r_1$, $\vec r_2$, and $\vec r_3$ be the rows of the coefficient matrix
for system \eqref{EQREGSYSb}. Then the system is equivalent to
\[
	\mat{1&2&-2\\2&1&-5\\1&-4&1}\mat{x\\y\\z}=
	\left[\begin{array}{c}\vec r_1\\\hline\vec r_2\\\hline\vec r_3\end{array}\right]\vec x=
	\mat{\vec r_1\cdot \vec x\\\vec r_2\cdot\vec x\\\vec r_3\cdot \vec x}
	     =\mat{-15\\-21\\18}.
\]
In other words, we can interpret solutions to system \eqref{EQREGSYSb} as vectors whose dot product
with $\vec r_1$ is $-15$, whose dot product with $\vec r_2$ is $-12$, and whose dot product with
$\vec r_3$ is $18$. Given that the dot product has a geometric interpretation, this perspective will
sometimes be enlightening.

\Heading{Interpreting Homogeneous Systems}
Consider the homogeneous system/matrix equation
\begin{equation}
	\label{EQREGSYSc}
	\sysdelim..
		\systeme{
			x+2y-2z=0,
			2x+y-5z=0,
			x-4y+z=0
		}
		\qquad\iff\qquad
		\underbrace{\mat{1&2&-2\\2&1&-5\\1&-4&1}}_A\mat{x\\y\\z}=\mat{0\\0\\0}
		.
\end{equation}
Now, the column interpretation of system \eqref{EQREGSYSc} is: what linear combinations of the columns vectors
of $A$ give $\vec 0$? This directly relates to the question of whether the column vectors of $A$ are linearly independent.

Let $\vec r_1$, $\vec r_2$, and $\vec r_3$ be the rows of $A$. The row interpretation of system \eqref{EQREGSYSc} asks
what vectors are simultaneously orthogonal to $\vec r_1$, $\vec r_2$, and $\vec r_3$.

\begin{emphbox}[Takeaway]
	There are three ways to interpret solutions to a matrix equation $A\vec x=\vec b$: (i) the intersection of hyperplanes
	specified by the rows; (ii) what linear combinations of the columns of $A$ give $\vec b$; (iii) what vectors yield
	the entries of $\vec b$ when dot producted with the rows of $A$.
\end{emphbox}

\begin{example}
	Find all vectors orthogonal to $\vec a=\mat{1\\1\\1}$ and $\vec b=\mat{1\\2\\1}$.

	To find all vectors orthogonal to $\vec a$ and $\vec b$, it is equivalent to solve
	\[
	\underbrace{\mat{1&1&1\\1&2&1}}_A\mat{x\\y\\z}=
	\left[\begin{array}{c}\vec a\\\hline\vec b\end{array}\right]\vec x=
	\mat{\vec a\cdot \vec x\\\vec b\cdot\vec x}
	     =\mat{0\\0}.
    \]
    By row reduction on matrix A as follows:
    \[
        \mat{1&1&1\\1&2&1} \sim \mat{1&1&1\\0&1&0} \sim \mat{1&0&1\\0&1&0},
    \]
    we obtain that the solution $\vec x$ need to satisfy:
    \begin{equation}
	\sysdelim\{.
		\systeme{
			x+z=0,
			y=0
		}.
    \end{equation}
    Therefore, the solution is $\vec x = t\mat{1\\0\\-1}$, for some $t\in \R$.
	
\end{example}

The row picture is particularly applicable when trying to find normal vectors.

\begin{example}
	Let $\mathcal Q$ be the hyperplane specified in vector form by
	\[
		\vec x=t\mat{1\\1\\-1\\1}+s\mat{0\\1\\0\\1}+r\mat{2\\0\\0\\0}+\mat{1\\2\\3\\4}.
	\]
	Find a normal vector for $\mathcal Q$ and write $\mathcal Q$ in normal form.

	Similarly to the above example, since normal vectors for $\mathcal Q$ need to be orthogonal to $\vec d_1=\mat{1\\1\\-1\\1}$, $\vec d_2=\mat{0\\1\\0\\1}$ and $\vec d_3=\mat{2\\0\\0\\0}$, we could find the normal vectors by solving
	\[
	\underbrace{\mat{1&1&-1&1\\0&1&0&1\\2&0&0&0}}_A\mat{x\\y\\z\\w}=\mat{0\\0\\0}.
    \]
    By row reduction on matrix A as follows:
    \begin{align*}
        \mat{1&1&-1&1\\0&1&0&1\\2&0&0&0} 
        &\sim \mat{2&0&0&0\\0&1&0&1\\1&1&-1&1}\sim \mat{1&0&0&0\\0&1&0&1\\0&1&-1&1} \\
        &\sim \mat{1&0&0&0\\0&1&0&1\\0&0&-1&0}\sim \mat{1&0&0&0\\0&1&0&1\\0&0&1&0},
    \end{align*}
    we obtain that the solution $\vec x$ need to satisfy:
    \begin{equation}
	\sysdelim\{.
		\systeme{
		    x=0,
			y+w=0,
			z=0
		}.
    \end{equation}
    Therefore, one normal vector can be $\vec n =\mat{0\\1\\0\\-1}$, and the normal form of $\mathcal Q$ is 
    \[
        \vec n\cdot(\vec x-\vec p)=\vec 0
    \]
    where $\vec p=\mat{1\\2\\3\\4}$.
\end{example}



