\begin{exercises}
	\begin{problist}
		\prob  Find the range and null space of the following linear transformations.
		\begin{enumerate}
		    \item   Let $\mathcal P:\R^2\to\R^2$, where $\mathcal P$ is projection on to the line $y=ax+b$, for some $a,b \in R$.
			\item   Let $\theta\in \R$ and let $\mathcal R:\R^2\to\R^2$ to be the transformation which rotates all vectors by $\theta$.
			\item   Let $\mathcal F:\R^2\to\R^2$, where $\mathcal F$ reflects over the line $y=ax+b$, for some $a,b \in R$.
			\item   Let $k\in R$ be non-zero and let $\mathcal S:\R^2\to\R^2$ which stretches all vectors by a factor of $k$.
		\end{enumerate}
		
		\prob For the following matrices, find their null space, column space, and row space.
		    \begin{enumerate}
		        \item $M_1=\mat{1&2&1\\3&1&-2\\8&6&-2}$.
		        \item $M_2=\mat{0&2&1\\3&2&5}$.
		        \item $M_3=\mat{1&2\\3&1\\4&0}$.
		        \item $M_4=\mat{1&-2&0&-1\\3&5&-1&0\\2&3&-2&0\\0&0&0&1}$.
		    \end{enumerate}
		
		\prob Given the transformation $\mathcal T$ and the vector $\vec V$ as follows, compute the $[\mathcal T_{M}\vec v]_{\mathcal E}$ and $\mathcal T(\vec v)$ for each question.
		\begin{enumerate}
		    \item Let $\mathcal T$ be the transformation induced by the matrix $M=\mat{7&5\\-2&-2}$,
	    and $\vec v=3\xhat-3\yhat$.
	        \item Let $\mathcal T$ be the transformation induced by the matrix $M=\mat{3&7&5\\1&-2&-2}$,
	    and $\vec v=2\xhat+0\yhat+4\zhat$.
		\end{enumerate}
		
		\prob Let $\mathcal P$ be the plane given by $3x+4y+5z=0$, and let $T:\R^3\to\R^3$ be projection onto $\mathcal P$. 
		\begin{enumerate}
		    \item Find $\Range(T)$ and $\Rank(T)$.
		    \item Find $\Null(T)$ and $\Nullity(T)$.
		\end{enumerate}
		\prob For each statement listed below, indicate whether it is correct or incorrect. Justify you answer.
		\begin{enumerate}
		    \item Let $A$ be an arbitrary matrix. Then $\Range(A)=\Range(A^T)$.
		    \item Let $T:\R^m\to\R^n$ be a transformation (not necessarily linear). If $\Set{x \given T(x)=0} \subseteq \R^m$ is a subspace, then $T$ is linear.
		    \item Let $T:\R^m\to\R^n$ be a linear transformation. Then $\Nullity(T) \geq n$.
		    \item Let $T:\R^m\to\R^n$ be a linear transformation induced by matrix $M$. Further, assume $\Rank(T) = n$, then $\Nullity(M) = 0$.
		\end{enumerate}

	\end{problist}
\end{exercises}
