\begin{exercises}
	\begin{problist}
		\prob For each linear transformation defined below, find its
		eigenvectors and eigenvalues. If it has no eigenvectors/values,
		explain why not.
		\begin{enumerate}
			\item $\mathcal S:\R^{2}\to\R^{2}$, where $\mathcal S$
				stretches every
				vector by the factor of $3$.

			\item $\mathcal R:\R^{2}\to\R^{2}$, where $\mathcal R$  
				rotates every vector
				clockwise by $\frac{\pi}{4}$.

			\item $\mathcal P:\R^{2}\to\R^{2}$, where $\mathcal P$
				projects every vector
				onto the line $\ell$ given by $y=-x$.

			\item $\mathcal F:\R^{2}\to\R^{2}$, where $\mathcal F$
				reflects every
				vector over the line $\ell$ given by $y=-x$.

			\item $T:\R^{3}\to\R^{3}$, where $T$ is a linear
				transformation induced by the matrix
				$\mat{1&2&3\\3&4&5\\5&6&7}$.

			\item $U:\R^{3}\to\R^{2}$, where $U$ is a linear
				transformation induced by the matrix
				$\mat{1&2&3\\3&4&5}$.
		\end{enumerate}

		\prob Let $A = \mat{a&b\\c&d}$, where $a,b,c,d \in \R$.
		\begin{enumerate}
			\item Find the characteristic polynomial of $A$.
		
			\item Find conditions on $a,b,c,d$ so that $A$ has (i) two distinct
				real eigenvalues, (ii) exactly one real eigenvalue, (iii) no
				real eigenvalues.
		\end{enumerate}

		\prob Let $B=\mat{1&2\\0&4}$.
		\begin{enumerate}
			\item Find the eigenvalues of $B$.
			\item Find the eigenvalues of $B^T$.
			\item A vector $\vec v\neq\vec 0$ is called a \emph{left-eigenvector} for $B$ if
				$\vec vB=\lambda \vec v$ for some scalar $\lambda$ (Here we consider $\vec v$ a \emph{row} vector). Find all
				left eigenvectors for $B$.
		\end{enumerate}

		\prob For each statement below, determine whether it is true or false. Justify you answer.
		\begin{enumerate}
			\item Zero cannot be an eigenvalue of any matrix.
			\item $\vec 0$ cannot be an eigenvector of any matrix.
			\item A $2\times 2$ matrix always has a real eigenvalue.
			\item A $3\times 3$ matrix always has a real eigenvalue.
			\item A $3\times 2$ matrix always has a real eigenvalue.
			\item The matrix $M = \mat{3&3&3&3\\3&3&3&3\\3&3&3&3\\3&3&3&3}$
				has exactly one eigenvalue.
			\item An invertible matrix can never have zero as an eigenvalue.
		\end{enumerate}
	\end{problist}
\end{exercises}
