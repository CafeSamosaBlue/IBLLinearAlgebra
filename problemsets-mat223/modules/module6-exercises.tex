\begin{exercises}
	% Topics
	% formal + intuitive definition of subspace
	% relationship between subspace and spans
	% prove a set is a subspace
	% basis + dimension of a subspace
	\begin{problist}

		\prob For each of the following sets, prove whether or not it is a subpace.
		\begin{enumerate}
			\item  $\mathcal T \subseteq \R^2$, where $\mathcal T$ is the complete solution to $3x-y=0$.
			\item  $\mathcal U \subseteq \R^2$, where $\mathcal U$ is the complete solution to $\frac{1}{2}x-6y=0$.
			\item  $\mathcal V \subseteq \R^2$, where $\mathcal V$ is the complete solution to $x-5y-1=0$.
			\item  $\mathcal X \subseteq \R^3$, where $\mathcal X$ is the complete solution to
			$5x-\pi y + (\ln 2)z=0$.
			\item $\mathcal Q \subseteq \R^n$, where $\mathcal Q$ is the complete solution to
			$a_1x_1+a_2x_2+...+a_{n-1}x_{n-1}+a_nx_n=0$ where $a_1,...,a_n \in \R$.
		\end{enumerate}

		\prob For each of the following sets, prove whether or not it is a subpace.
		\begin{enumerate}
			\item $\mathcal A \subseteq \R^2$, where $\mathcal A$ is the the line specified in vector form by
			$\vec x = t\mat{5\\-7}+\mat{1\\2}$.
			\item $\mathcal B \subseteq \R^2$, where $\mathcal B$ is  the the line specified  in vector form by
			$\vec x = t\mat{-3\\4}$.
			\item $\mathcal C \subseteq \R^3$, where $\mathcal A$ is  the the line specified  in vector form by
			$\vec x = t\mat{1\\0\\5}$.
			\item $\mathcal D \subseteq \R^3$, where $\mathcal A$ is  the the line specified  in vector form by
			$\vec x = t\mat{2\\3\\4}+s\mat{10\\20\\131}+\mat{0\\0\\6}$.
			\item $\mathcal E \subseteq \R^3$, where $\mathcal A$ is  the the line specified  in vector form by
			$\vec x = t\mat{5\\7\\1}+s\mat{2\\-2\\1}$.
		\end{enumerate}

		\prob Use the definition of subspace to prove each span below is a subspace.
		\begin{enumerate}
			\item $\Span\Set*{\mat{0\\1}, \mat{1\\2}}$
			\item $\Span\Set*{\mat{1\\1\\1}, \mat{1\\0\\0}, \mat{2\\0\\0}}$
		\end{enumerate}
		
		\prob
		A non-empty subset $\mathcal V \subseteq \R^n$ is called a subspace if
		for all $\vec u, \vec v \in \mathcal V$ and all scalars $k$ we have
			(i) $\vec u + \vec v \in \mathcal V$ and
			(ii) $k\vec u \in \mathcal V$.
			For each set below, list which of property (i), property (ii), or non-emptiness fails.
			Justify your answer.
		\begin{enumerate}
			\item $\Set*{(x,y,z)\given x+y+z=4}$
			\item $\Set*{}$
			\item $\Set*{(x,y)\given x=y^2}$
			\item $\Set*{(x_1,x_2)\given x_1 \geq 0}$
			\item $\Set*{(x,y)\given x^2+y^2=0}$
		\end{enumerate}

		\prob For each subspace below, determine whether or not it is a subspace.
		If it is a subspace, find (i) its dimension and (ii) a basis for it.
		\begin{enumerate}
			\item $\Span\Set*{\mat{2\\3}, \mat{-4\\-6}, \matc{1\\3/2}}$
			\item $\Span\Set*{\mat{1\\0\\-2}, \mat{0\\2\\5}, \mat{1\\2\\3}}$
			\item The plane given in vector form by
			\[
				\vec x = t\mat{6\\1\\1}+s\mat{6\\0\\6}
			\]
			\item The line given in vector form by
			\[
				\vec x = t\mat{2\\2\\3}
			\]
			\item The complete solution to
			\[
				\mat{1\\-2\\3} \cdot \left(\mat{x\\y\\z}-\mat{2\\2\\\frac{2}{3}}\right)=0
			\]
			\item The complete solution to
			\[
				\mat{1\\3\\3\\7} \cdot \left(\mat{x\\y\\z\\w}-\mat{0\\0\\0\\0}\right)=0
			\]
		\end{enumerate}

		\prob Which of the following are bases for $\R^3$?
		\begin{enumerate}
			\item $\Set*{\mat{2\\6\\1}, \mat{4\\2\\1}, \mat{6\\8\\2}}$
			\item $\Set*{\mat{1\\0\\1}, \mat{1\\0\\0}, \mat{0\\-1\\0}, \mat{-2\\1\\1}}$
			\item $\Set*{\mat{2\\3\\5}, \mat{5\\-4\\2}}$
			\item $\Set*{\mat{2\\5\\-6}, \mat{4\\11\\-12}, \mat{0\\0\\-3}}$
		\end{enumerate}


		\prob  For each statement, determine if it is true or false.
		Justify your answer by referring to a definition or a theorem.
		\begin{enumerate}
			\item All spans are subspaces.
			\item All subspaces can be expressed as spans.
			\item All translated spans are subspaces.
			\item The empty set is a subspace.
			\item The set $\Set*{\mat{1\\2}, \mat{2\\3}}$ is a subspace.
		\end{enumerate}

		\prob Give two examples of subspaces of $\R^4$ that are (i) 1 dimensional, (ii)
		3 dimensional. Can you give an example of a subspace that is 0 dimensional? 

		\prob Let $\mathcal G\subseteq \R^n$ be a subspace.
		Give upper and lower bounds for the dimension of $\mathcal G$.
	\end{problist}
\end{exercises}
