\begin{exercises}
	% Topics
	% formal + intuitive definition of subspace
	% relationship between subspace and spans
	% prove a set is a subspace
	% basis + dimension of a subspace
	\begin{problist}
		\prob  State the definition of a subspace.

		\prob  True of False:
		\begin{enumerate}
			\item All spans are subspaces.
			\item All subspaces can be expressed as spans.
			\item All translated spans are subspaces.
			\item The empty set is a subspace.
			\item The set $\Set*{\mat{1\\2}, \mat{2\\3}}$ is a subspace.
		\end{enumerate}

		\prob If the following is a subspace, prove it. Otherwise, explain why not.
		\begin{enumerate}
			\item Let $\mathcal T \subseteq \R^2$ be the complete solution to $3x-y=0$.
			\item Let $\mathcal U \subseteq \R^2$ be the complete solution to $\frac{1}{2}x-6y=0$.
			\item Let $\mathcal V \subseteq \R^2$ be the complete solution to $x-5y-1=0$.
			\item Let $\mathcal W \subseteq \R^3$ be the complete solution to $x+2y-3z=0$.
			\item Let $\mathcal X \subseteq \R^3$ be the complete solution to
			$ex-\pi y + z\ln 2=0$.
			\item Let $n \in \N$. Let $a_1,...,a_n \in \R$. Let $\mathcal Q \subseteq \R^n$ be the complete solution to $a_1x_1+a_2x_2+...+a_{n-1}x_{n-1}+a_nx_n=0$.
		\end{enumerate}

		\prob If the following is a subspace, prove it. Otherwise, explain why not.
		\begin{enumerate}
			\item Let $\mathcal B \subseteq \R^2$ be the the line in vector form
			$\vec x = t\mat{5\\-7}+\mat{1\\2}$
			\item Let $\mathcal A \subseteq \R^2$ be the the line in vector form
			$\vec x = t\mat{-3\\4}$
			\item Let $\mathcal C \subseteq \R^3$ be the the line in vector form
			$\vec x = t\mat{1\\0\\5}$
			\item Let $\mathcal D \subseteq \R^3$ be the the line in vector form
			$\vec x = t\mat{2\\3\\4}+s\mat{10\\20\\131}+\mat{0\\0\\6}$
			\item Let $\mathcal E \subseteq \R^3$ be the the line in vector form
			$\vec x = t\mat{5\\7\\1}+s\mat{2\\-2\\1}$
		\end{enumerate}

		\prob Show that the following spans are subspaces.
		\begin{enumerate}
			\item $\Span\Set*{\mat{-1\\1}}$
			\item $\Span\Set*{\mat{0\\1}, \mat{1\\2}}$
			\item $\Span\Set*{\mat{1\\1\\1}, \mat{1\\0\\0}, \mat{2\\0\\0}}$
			\item $\Span\Set*{\mat{5\\4\\3\\2}}$
		\end{enumerate}

		\prob For each set:
		\begin{itemize}
			\item[-] find the dimension.
			\item[-] find 2 bases.
		\end{itemize}
		\begin{enumerate}
			\item $\Span\Set*{\mat{2\\3}, \mat{-4\\-6}, \mat{1\\\frac{3}{2}}}$
			\item $\Span\Set*{\mat{1\\4}, \mat{3\\-1}, \mat{2\\8}}$
			\item $\Span\Set*{\mat{1\\0\\-2}, \mat{0\\2\\5}, \mat{1\\2\\3}}$
			\item $\Span\Set*{\mat{2\\0\\1}, \mat{0\\3\\0}, \mat{0\\0\\4}}$
		\end{enumerate}

		\prob Let $\mathcal V$ be a subspace. What properties must a set $\mathcal B$ satisfy
		in order to be a basis for $\mathcal V$.

		\prob Give 2 examples of subspaces of $\R^4$ that are 1 dimensional, 2 dimensional,
		3 dimensional, and 4 dimensional.

		\prob Let $n \in \N$. Let $\mathcal G$ be a subspace of $\R^n$. Is there a
		smallest dimension $\mathcal G$ can have? A largest dimension?
	\end{problist}
\end{exercises}
