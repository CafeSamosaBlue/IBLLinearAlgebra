With a handle on vectors, we can now use them to describe some common geometric
objects: lines and planes.

\Heading{Lines}
Consider for a moment the line $\ell$ through the points $P$ and $Q$.  When $P,Q\in\R^2$, we
can describe $\ell$ with an equation of the form $y=mx+b$ (provided it isn't a vertical line), but if
$P,Q\in\R^3$, it's much harder to describe $\ell$ with an equation.  Using vectors
provides an easier way.

Let $\vec d=\overrightarrow{PQ}$ and consider the set of points (or vectors) $\vec x$ that can be expressed as
\[
	\vec x=t\vec d+P
\]
for $t\in \R$.  Geometrically, this is the set of all points we get by starting at $P$ and
displacing by some multiple of $\vec d$.  This is a line!


\begin{center}
	\begin{tikzpicture}
		\coordinate (A) at (1,1);
		\coordinate (B) at (3,2);
		\coordinate (D) at ($(B)-(A)$);
		\begin{axis}[
		    anchor=origin,
		    disabledatascaling,
		    xmin=-1,xmax=5,
		    ymin=-1,ymax=3,
		    x=1cm,y=1cm,
		    grid=both,
		    grid style={line width=.1pt, draw=gray!10},
		    %major grid style={line width=.2pt,draw=gray!50},
		    axis lines=middle,
		    minor tick num=0,
		    enlargelimits={abs=0.5},
		    axis line style={latex-latex},
		    ticklabel style={font=\tiny,fill=white},
		    xlabel style={at={(ticklabel* cs:1)},anchor=north west},
		    ylabel style={at={(ticklabel* cs:1)},anchor=south west}
		]

		\draw [Green,fill] (A) circle (1.5pt) node [below right] {$P$};
		\draw [Green,fill] (B) circle (1.5pt) node [below right] {$Q$};
		\draw[->,thick,myred!60!white] (A) -- (B) node [midway,below right,yshift=2pt] {$\vec d$};


		\end{axis}
		\foreach \x in {-1,5} {
			\draw [mypink,fill] ($(A)+\x/3*(D)$) circle (1.5pt) node [left] {\footnotesize$\tfrac{\x}{3}\vec d+P$};
		}
		\foreach \x in {-3,-2,4,6} {
			\draw [mypink,fill] ($(A)+\x/3*(D)$) circle (1.5pt);
		}
	\end{tikzpicture}
	\begin{tikzpicture}
		\coordinate (A) at (1,1);
		\coordinate (B) at (3,2);
		\coordinate (D) at ($(B)-(A)$);
		\begin{axis}[
		    anchor=origin,
		    disabledatascaling,
		    xmin=-1,xmax=5,
		    ymin=-1,ymax=3,
		    x=1cm,y=1cm,
		    grid=both,
		    grid style={line width=.1pt, draw=gray!10},
		    %major grid style={line width=.2pt,draw=gray!50},
		    axis lines=middle,
		    minor tick num=0,
		    enlargelimits={abs=0.5},
		    axis line style={latex-latex},
		    ticklabel style={font=\tiny,fill=white},
		    xlabel style={at={(ticklabel* cs:1)},anchor=north west},
		    ylabel style={at={(ticklabel* cs:1)},anchor=south west}
		]

		\draw [Green,fill] (A) circle (1.5pt) node [below right] {$P$};
		\draw [Green,fill] (B) circle (1.5pt) node [below right] {$Q$};
		\draw[->,thick,myred!60!white] (A) -- (B) node [midway,below right,yshift=2pt] {$\vec d$};

		\end{axis}

		\foreach \x in {-3,-2,-1,0,4,5,6} {
			\draw [->, gray!50!white] (0,0) -- ($(A)+\x/3*(D)$);
		}
		\foreach \x in {-1,5} {
			\draw [mypink,fill] ($(A)+\x/3*(D)$) circle (1.5pt) node [left] {\footnotesize$\tfrac{\x}{3}\vec d+P$};
		}
		\foreach \x in {-3,-2,4,6} {
			\draw [mypink,fill] ($(A)+\x/3*(D)$) circle (1.5pt);
		}
	\end{tikzpicture}
\end{center}
We simultaneously interpret this line as a set of points (the points that make up
the line) and as a set of vectors 
rooted at the origin (the vectors pointing from the origin to the line).
Note that sometimes we draw vectors as directed line segments.
Other times, we draw
each vector by marking only its ending point because drawing each vector as line segment
would make
it hard to see what is going on.

Which picture below do you think best represents $\ell$?
\begin{center}
	\begin{tikzpicture}
		\coordinate (A) at (1,1);
		\coordinate (B) at (3,2);
		\coordinate (D) at ($(B)-(A)$);
		\begin{axis}[
		    anchor=origin,
		    disabledatascaling,
		    xmin=-1,xmax=5,
		    ymin=-1,ymax=3,
		    x=1cm,y=1cm,
		    grid=both,
		    grid style={line width=.1pt, draw=gray!10},
		    %major grid style={line width=.2pt,draw=gray!50},
		    axis lines=middle,
		    minor tick num=0,
		    enlargelimits={abs=0.5},
		    axis line style={latex-latex},
		    ticklabel style={font=\tiny,fill=white},
		    xlabel style={at={(ticklabel* cs:1)},anchor=north west},
		    ylabel style={at={(ticklabel* cs:1)},anchor=south west}
		]

		\node[above right] at (0,3) {~Vectors in $\ell$ as dots};
		\draw [mypink, thick] ($(A)-3*(D)$) -- ($(A)+3*(D)$);
		\end{axis}
	\end{tikzpicture}
	\begin{tikzpicture}
		\coordinate (A) at (1,1);
		\coordinate (B) at (3,2);
		\coordinate (D) at ($(B)-(A)$);
		\begin{axis}[
		    anchor=origin,
		    disabledatascaling,
		    xmin=-1,xmax=5,
		    ymin=-1,ymax=3,
		    x=1cm,y=1cm,
		    grid=both,
		    grid style={line width=.1pt, draw=gray!10},
		    %major grid style={line width=.2pt,draw=gray!50},
		    axis lines=middle,
		    minor tick num=0,
		    enlargelimits={abs=0.5},
		    axis line style={latex-latex},
		    ticklabel style={font=\tiny,fill=white},
		    xlabel style={at={(ticklabel* cs:1)},anchor=north west},
		    ylabel style={at={(ticklabel* cs:1)},anchor=south west}
		]
			\node[above right] at (0,3) {~Vectors in $\ell$ as arrows from $(0,0)$};
		\end{axis}

		\foreach \x in {-8,...,14} {
			\draw [->, mypink] (0,0) -- ($(A)+\x/6*(D)$);
		}
	\end{tikzpicture}
\end{center}


\begin{emphbox}[Takeaway]
	When drawing a picture depicting several vectors, make an appropriate choice (arrows, dots, or
	a mix) so that the picture is clear.
\end{emphbox}

The line $\ell$ described above can be written in set-builder notation
as:
\[
	\ell=\Set{\vec x\given \vec x=t\vec d+P\text{ for some }t\in \R}.
\]
Notice that in set-builder notation, we write ``for some $t\in \R$.'' Make sure you
understand why replacing ``for some $t\in\R$'' with ``for
all $t\in \R$'' would be incorrect.

\medskip
Writing lines with set-builder notation all the time can be overkill,
so we will allow ourselves to describe lines in a shorthand called \emph{vector form}\footnote{
	$y=mx+b$ form of a line is also shorthand.  The line $\ell$ described by the equation
	$y=mx+b$ is actually the set $\Set{(x,y)\in\R^2\given y=mx+b}$.
}.

\SavedDefinitionRender{VectorFormofaLine}

We can also use coordinates when writing a line in vector form. For example,
\[
	\mat{x\\y}=t\mat{d_1\\d_2}+\mat{p_1\\p_2}
\]
corresponds to the line passing through $\mat{p_1\\p_2}$ with $\mat{d_1\\d_2}$ as a direction vector.

The ``$t$'' that appears in a vector form is called the \emph{parameter variable}, and for this
reason, some textbooks use the term \emph{parametric form} in place of ``vector form''. 

\medskip
Writing a line in vector form requires only a point on the line and a direction for the line\footnote{ Notice that
a direction vector \emph{for} a line $\ell$ is different than a vector \emph{in} a line $\ell$.},
which makes converting from another form into vector form straightforward.
\begin{example}
	Find vector form of the line $\ell\subseteq\R^2$ with equation $y=2x+3$.

	First, we find two
	points on $\ell$.  By guess-and-check, we see $P=(0,3)$ and $Q=(1,5)$ are on $\ell$.
	Thus, a direction vector for $\ell$ is given by
	\[
		\vec d = \mat{1\\5}-\mat{0\\3}=\mat{1\\2}.
	\]
	We may now express $\ell$ in vector form as
	\[
		\vec x=t\vec d+P
	\]
	or, using coordinates,
	\[
		\mat{x\\y} = t\mat{1\\2}+\mat{0\\3}.
	\]
\end{example}

It's important to note that when we write a line in vector form, it is a \emph{specific shorthand} notation.
If we augment the notation, we no longer have written a line in ``vector form''.

\begin{example} Let $\ell$ be a line, let $\vec d$ be a direction vector for $\ell$, and let $\vec p\in \ell$
	be a point on $\ell$. Writing
	\[
		\vec x=t\vec d+\vec p
	\]
	or 
	\[
		\vec x=t\vec d+\vec p\quad\text{ where }\quad t\in \R
	\]
	specifies $\ell$ in vector form; both are shorthands for $\Set{\vec x\given
	\vec x=t\vec d+\vec p\text{ for some }t\in \R}$. But,
	\[
		\vec x=t\vec d+\vec p\quad\text{ for some }\quad t\in \R
	\]
	and
	\[
		\vec x=t\vec d+\vec p\quad\text{ for all }\quad t\in \R
	\]
	are logical statements about the vectors $\vec x$, $\vec d$, and
	$\vec p$. These statements are either true or false; they do \emph{not} 
	specify $\ell$ in vector form.

	Similarly, the statement
	\[
		\ell = t\vec d+\vec p
	\]
	is mathematically nonsensical and does not specify $\ell$ in vector form. (On the
	left is a \emph{set} and on the right is a \emph{vector}!)

\end{example}

\begin{emphbox}[Takeaway]
	Vector form is a specific shorthand for a set. If ``extra'' words or symbols are added
	to the vector form, it stops being a shorthand.
\end{emphbox}

But, why is vector form useful? For starters, every line can be expressed in vector form 
(you cannot write a vertical line in $y=mx+b$ form, and in $\R^3$, you would need two linear equations
to represent a line). But, the most useful thing about expressing a line in vector form is that
you can easily generate points on that line.

Suppose $\ell$ can be represented in vector form as $\vec x=t\vec d+\vec p$.
Then, for every $t\in \R$, the vector $t\vec d +\vec p\in\ell$. Not only that, but as $t$ ranges over
$\R$, all points on $\ell$ are ``traced out''. Thus, we can find points on $\ell$ without having to ``solve''
any equations.

The downside to using vector form is that it is not unique. There are multiple direction vectors and multiple points
for every line.  Thus, merely by looking at the vector equation for two lines, it can be hard to tell if
they're equal.

For example,
\[
	\mat{x\\y} = t\mat{1\\2}+\mat{0\\3},\qquad
	\mat{x\\y} = t\mat{2\\4}+\mat{0\\3},\quad\text{and}\quad
	\mat{x\\y} = t\mat{1\\2}+\mat{1\\5}
\]
all represent the same line.  In the second equation, the direction vector is parallel but scaled, and in
the third equation, a different point on the line was chosen.

Recall that in vector form, the variable $t$ is called the \emph{parameter variable}.  It is an instance of
a \emph{dummy variable}. In other words, $t$ is a placeholder---just because ``$t$'' appears in two different vector
forms, doesn't mean it's the same quantity.

To drive this point home, let's think about vector form in terms
of the sets it specifies. Let $\vec d_1,\vec d_2\neq\vec 0$ and $\vec p_1,\vec p_2$ be vectors and define the lines
\[
	\ell_1=\Set{\vec x\given \vec x=t\vec d_1+\vec p_1\text{ for some }t\in\R}
\]
\[
	\ell_2=\Set{\vec x\given \vec x=t\vec d_2+\vec p_2\text{ for some }t\in\R}.
\]
These lines have vector forms $\vec x=t\vec d_1+\vec p_1$ and $\vec x=t\vec d_2+\vec p_2$.
However, declaring that $\ell_1=\ell_2$ if and only if $t\vec d_1+\vec p_1=t\vec d_2+\vec p_2$
does \emph{not} make sense.   Instead, as per the definition, $\ell_1=\ell_2$ if $\ell_1\subseteq\ell_2$ and $\ell_2\subseteq\ell_1$.
If $\vec x\in\ell_1$ then $\vec x=t\vec d_1+\vec p_1$ for some $t\in\R$.  If $\vec x\in\ell_2$
then $\vec x=t\vec d_2+\vec p_2$ for some \emph{possibly different} $t\in \R$.  This can get
confusing really quickly.  The easiest way to avoid confusion is to use different parameter variables 
when comparing different vector forms.

\begin{example}
	Determine if the lines $\ell_1$ and $\ell_2$, given in vector form as 
	\[
		\vec x=t\mat{1\\1}+\mat{2\\1}\qquad\text{and}\qquad
		\vec x=t\mat{2\\2}+\mat{4\\3},
	\]
	are the same line.  
	
	To determine this, we need to figure out if $\vec x\in\ell_1$
	implies $\vec x\in \ell_2$ and if $\vec x\in\ell_2$ implies $\vec x\in\ell_1$.

	If $\vec x\in\ell_1$, then $\vec x=t\mat{1\\1}+\mat{2\\1}$ for some $t\in\R$.  If
	$\vec x\in\ell_2$, then $\vec x=s\mat{2\\2}+\mat{4\\3}$ for some $s\in \R$.  Thus if
	\[
		t\mat{1\\1}+\mat{2\\1} = \vec x = s\mat{2\\2}+\mat{4\\3}
	\]
	always has a solution, $\ell_1=\ell_2$.  Moving everything to one side, we see
	\begin{align*}
		\vec 0 = \mat{4\\3}-\mat{2\\1} + s\mat{2\\2}-t\mat{1\\1}
		&=\mat{2\\2}+s\mat{2\\2}-t\mat{1\\1}\\
		&=(s+1)\mat{2\\2}-\tfrac{t}{2}\mat{2\\2}\\
		&= (s+1-\tfrac{t}{2})\mat{2\\2}.
	\end{align*}
	This has a solution whenever $s+1-t/2=0$.  Since for every $t\in\R$ we can find an $s\in\R$ satisfying
	this equation
	and for every $s\in \R$ we can find a $t\in \R$ satisfying this equation, we know $\ell_1=\ell_2$.
\end{example}


\Heading{Vector Form in Higher Dimensions}
The geometry of lines in space ($\R^3$ and above) is a bit more complicated than that of lines
in the plane.  Lines in the plane either intersect or are parallel.
In space,  we have to be a more careful about what we mean by
``parallel lines,'' since lines with entirely different directions can
still fail to intersect\footnote{ Recall that in Euclidean geometry
two lines are defined to be parallel if they coincide or never intersect.}.

\begin{center}
  \begin{tikzpicture}
    \begin{axis}[grid=major,view={20}{40},z buffer=sort,
	    zmin=0,
	    xticklabels={,,}, yticklabels={,,}, zticklabels={,,}
	    ]
	    \addplot3 [no marks,orange,ultra thick] coordinates {(0,10,10) (20,10,30)};
	    \addplot3 [no marks,orange,dashed, thick] coordinates {(0,10,0) (20,10,0)};
	    \addplot3 [no marks,mypink,ultra thick] coordinates {(0,0,20) (20,20,0)};
	    \addplot3 [no marks,mypink,dashed, thick] coordinates {(0,0,0) (20,20,0)};
	%\addplot3[domain=4:30,samples=80,samples y=0,mark=none,orange, opacity=0.5,ultra thick]
	%    ({x},{118.89/x},{2*x});
    \end{axis}
  \end{tikzpicture}
\end{center}

\begin{example}
Consider the lines described by
\begin{align*}
\vec x &= t( 1, 3, -2 ) + ( 1, 2, 1 ) \\
\vec x &= t( -2, -6, 4) + ( 3, 1, 0 ).
\end{align*}
They have parallel directions since $( -2, -6, 4 ) = -2( 1, 3,-2 )$.
Hence, in this case, we say the lines are \emph{parallel}\index{parallel lines}.  (How can
we be sure the lines are not the same?)
\end{example}

\begin{example}
Consider the lines described by
\begin{align*}
	\vec x &= t(1, 3, -2 ) + ( 1, 2, 1 ) \\
	\vec x &= t( 0, 2, 3) + ( 0, 3, 9 ).
\end{align*}
They are not parallel because neither of the direction
vectors is a multiple
of the other.  They may or may not intersect.  (If they don't,
	we say the lines are \emph{skew}\index{skew lines}.)  How can we find out?
	Mirroring our earlier approach,
	we can set their equations equal and see if we can solve for a point
	of intersection \emph{after ensuring we give their parametric variables
	different names}.   We'll keep one parametric variable named $t$ and name the
	other one $s$.  Thus, we want
\[
\vec x = t( 1, 3, -2 ) + ( 1, 2, 1 ) =
s( 0, 2, 3) + ( 0, 3, 9 ),
\]
which after collecting terms yields
\[
    ( t + 1, 3t + 2, -2t + 1 ) = ( 0, 2s + 3, 3s + 9).
\]
Reading coordinate by coordinate, we get three equations
\begin{align*}
    t + 1 &= 0 \\
    3t +2 &= 2s + 3 \\
    -2t + 1 &=  3s + 9
\end{align*}
in two unknowns  $s$ and $t$.  This is an {\it overdetermined\/}
system, and it may or may not have a solution.
The first two equations yield $t = -1$  and $s = -2$.  Putting
these values in the last equation yields $(-2)(-1) + 1 = 3(-2) + 9$,
which is indeed true.
Hence, the equations are consistent, and the lines
intersect.   To find the point of intersection, put $t = -1$
in the equation for the vector equation of the first line (or
$s = -2$ in that for the second) to obtain  $( 0, -1, 3 )$.
\end{example}

\Heading{Planes}

Any two distinct points define a line.  To define a plane, we
need three points.  But there's a caveat: the three points cannot
be on the same line, otherwise they'd define a line
and not a plane.  Let $A,B,C\in\R^3$ be three points that are not
collinear and let $\mathcal P$ be the plane that passes through $A$,
$B$, and $C$.

Just like lines, planes have direction vectors.  For $\mathcal P$, both
$\vec d_1=\overrightarrow{AB}$ and $\vec d_2=\overrightarrow{AC}$ are direction
vectors.  Of course, $\vec d_1$, $\vec d_2$ and their multiples
are not the only direction vectors for $\mathcal P$. There are infinitely many more, including
$\vec d_1+\vec d_2$, and $\vec d_1-7\vec d_2$, and so on.  However, since a plane
is a \emph{two}-dimensional object, we only need two different direction vectors to describe it.

Like lines, planes have a vector form.  Using the direction vectors $\vec d_1=\overrightarrow{AB}$ and 
$\vec d_2=\overrightarrow{AC}$,
the plane $\mathcal P$ can be written in vector form as
\[
	\mat{x\\y\\z} = t\vec d_1+s\vec d_2+A.
\]
\begin{center}
  \begin{tikzpicture}
    \begin{axis}[grid=major,view={20}{40},z buffer=sort,
	    %zmin=0,
	    xticklabels={,,}, yticklabels={,,}, zticklabels={,,}
	    ]
		\addplot3 [data cs=cart,surf,domain=-10:10,samples=2, opacity=0.5]
		{x+y};
		\coordinate (A) at (axis cs:-3,-3,-6);
		\coordinate (B) at (axis cs:3,4,7);
		\coordinate (C) at (axis cs:-4,4,0);

		\draw [mypink,fill] (A) circle (1.5pt) node [below right] {$A$};
		\draw [->, thick] (A) -- (B) node [midway,below right] {$\vec d_1$};
		\draw [->, thick] (A) -- (C) node [midway,above left] {$\vec d_2$};
    \end{axis}
  \end{tikzpicture}
\end{center}

\SavedDefinitionRender{VectorFormofaPlane}

\begin{example}
	Describe the plane $\mathcal P\subseteq \R^3$ with equation $z=2x+y+3$ in vector form.

	To describe $\mathcal P$ in vector form, we need a point on $\mathcal P$ and two direction
	vectors for $\mathcal P$. By guess-and-check, we see the points
	\[
		A=\mat{0\\0\\3}\qquad B=\mat{1\\0\\5}\qquad C=\mat{0\\1\\4}
	\]
	are all in $\mathcal P$. Thus
	\[
		\vec d_1=B-A=\mat{1\\0\\2}\qquad \text{and}\qquad
		\vec d_2=C-A=\mat{0\\1\\1}
	\]
	are both direction vectors for $\mathcal P$.  Since these vectors are not parallel, 
	we can express $\mathcal P$ in vector
	form as
	\[
		\vec x=t\vec d_1+s\vec d_2+A=t\mat{1\\0\\2}+s\mat{0\\1\\1}+\mat{0\\0\\3}.
	\]
\end{example}


\begin{example}
	Find the line of intersection between $\mathcal P_1$ and $\mathcal P_2$ where the planes
	are given in vector form by
	\[
		\overbrace{\vec x=t\mat{1\\1\\0}+s\mat{-1\\0\\1}+\mat{1\\2\\3}}^{\displaystyle\mathcal P_1}
		\qquad\text{and}\qquad
		\overbrace{\vec x=t\mat{-1\\0\\2}+s\mat{1\\2\\1}+\mat{0\\0\\3}}^{\displaystyle\mathcal P_2}.
	\]

	Just like in the example for lines, we are looking for points $\vec x$ that are in both planes. To
	keep from getting mixed, we'll use $a$, $b$, $c$, and $d$ as parameter variables. Therefore, we are looking
	for solutions to
	\[
		a\mat{1\\1\\0}+b\mat{-1\\0\\1}+\mat{1\\2\\3}=\vec x=c\mat{-1\\0\\2}+d\mat{1\\2\\1}+\mat{0\\0\\3}.
	\]
	Collecting terms, this is equivalent to the system of equations
	\[
		\systeme{a-b+c-d=-1,a-2d=-2,b-2c-d=0}.
	\]
	This system is underdetermined (there are four variables and three equations). If $\mathcal P_1$ 
	and $\mathcal P_2$ indeed intersect in a line, we know there must an infinite number
	of solutions to this system. After row reducing, we see
	\[
		\mat{a\\b\\c\\d} = \matc{r\\r/2-1\\-1\\r/2+1}
	\]
	is a solution for every $r\in\R$. We can substitute these parameters into either of the original equations
	to get an equation for the line of intersection. Picking the second one, we see
	\[
		\vec x=c\mat{-1\\0\\2}+d\mat{1\\2\\1}+\mat{0\\0\\3}=-\mat{-1\\0\\2}+(\tfrac{r}{2}+1)\mat{1\\2\\1}+\mat{0\\0\\3}
		=\tfrac{r}{2}\mat{1\\2\\1}+\mat{2\\2\\2}
	\]
	is in both planes for every $r\in \R$. Therefore, we may express $\mathcal P_1\cap \mathcal P_2$ in vector form as
	\[
		\vec x=r\matc{1/2\\1\\1/2}+\mat{2\\2\\2}.
	\]
\end{example}

\Heading{Restricted Linear Combinations}

Using vectors, we can describe more than just lines and planes---we can describe 
all sorts of geometric objects.

Recall that when we write $\vec x=t\vec d+\vec p$
to describe the line $\ell$,
what we mean is
\[
	\ell=\Set{\vec x\given \vec x=t\vec d+\vec p\text{ for some }t\in \R}.
\]
The line $\ell$ stretches off infinitely in both directions. But, what if we wanted
to describe just a part of $\ell$? We can do this by placing additional restrictions
on $t$.
For example, consider the ray $R$
and the line segment $S$:
\begin{align*}
	R&=\Set{\vec x\given \vec x=t\vec d+\vec p\text{ for some }t\geq 0}\\
	S&=\Set{\vec x\given \vec x=t\vec d+\vec p\text{ for some }t\in [0,2]}
\end{align*}

\begin{center}
	\begin{tikzpicture}
		\begin{axis}[
		    anchor=origin,
		    disabledatascaling,
		    xmin=-1,xmax=4,
		    ymin=-1,ymax=3,
		    x=1cm,y=1cm,
		    grid=both,
		    grid style={line width=.1pt, draw=gray!10},
			xtick={0,1,2,3,4},
		    %major grid style={line width=.2pt,draw=gray!50},
		    axis lines=middle,
		    minor tick num=0,
		    enlargelimits={abs=0.5},
		    axis line style={latex-latex},
		    ticklabel style={font=\tiny,fill=white},
		    xlabel style={at={(ticklabel* cs:1)},anchor=north west},
		    ylabel style={at={(ticklabel* cs:1)},anchor=south west}
		]

			\draw [mypink, very thick] (0,0) -- (5,3) node[midway, below right] {$R$};
			\draw [black!70!white, thick,dashed, ->, yshift=.15cm] (0,0) -- (5/5,3/5) node[midway, above] {$\vec d$};
		\end{axis}
			\node[above right] at (0,3) {~The ray $R$};
	\end{tikzpicture}
	~~~~
	\begin{tikzpicture}
		\begin{axis}[
		    anchor=origin,
		    disabledatascaling,
		    xmin=-1,xmax=4,
		    ymin=-1,ymax=3,
			xtick={0,1,2,3,4},
		    x=1cm,y=1cm,
		    grid=both,
		    grid style={line width=.1pt, draw=gray!10},
		    %major grid style={line width=.2pt,draw=gray!50},
		    axis lines=middle,
		    minor tick num=0,
		    enlargelimits={abs=0.5},
		    axis line style={latex-latex},
		    ticklabel style={font=\tiny,fill=white},
		    xlabel style={at={(ticklabel* cs:1)},anchor=north west},
		    ylabel style={at={(ticklabel* cs:1)},anchor=south west}
		]
			\draw [mygreen, very thick] (0,0) -- (10/5,6/5) node[midway, below right] {$S$};
			\draw [black!70!white, thick,dashed, ->, yshift=.15cm] (0,0) -- (5/5,3/5) node[midway, above] {$\vec d$};
		\end{axis}
			\node[above right] at (0,3) {~The line segment $S$};
	\end{tikzpicture}
\end{center}

We can also make polygons by adding restrictions to the vector form of a plane. Let
$\vec a=\mat{2\\1}$ and $\vec b=\mat{-1\\1}$ and consider the unit square $U$ and
the parallelogram $P$ defined by
\begin{align*}
	U&=\Set{\vec x\given \vec x=t\xhat+s\yhat\text{ for some }t,s\in [0,1]}\\
	P&=\Set{\vec x\given \vec x=t\vec a+s\vec b\text{ for some }t\in [0,1]\text{ and }s\in[-1,1]}
\end{align*}

\begin{center}
	\begin{tikzpicture}
		\begin{axis}[
		    anchor=origin,
		    disabledatascaling,
		    xmin=-2,xmax=3,
		    ymin=-1,ymax=2,
			xtick={-2,...,3},
			ytick={-2,...,3},
		    x=1cm,y=1cm,
		    grid=both,
		    grid style={line width=.1pt, draw=gray!10},
		    %major grid style={line width=.2pt,draw=gray!50},
		    axis lines=middle,
		    minor tick num=0,
		    enlargelimits={abs=0.5},
		    axis line style={latex-latex},
		    ticklabel style={font=\tiny,fill=white},
		    xlabel style={at={(ticklabel* cs:1)},anchor=north west},
		    ylabel style={at={(ticklabel* cs:1)},anchor=south west}
		]

			\fill [mypink, opacity=.3] (0,0) -- (1,0) -- (1,1) node[midway, right, opacity=1] {$U$} -- (0,1) -- cycle;
			\draw [black!70!white, thick,dashed, ->, yshift=-.15cm] (0,0) -- (1,0) node[midway, below] {$\xhat$};
			\draw [black!70!white, thick,dashed, ->, xshift=-.15cm] (0,0) -- (0,1) node[midway, left] {$\yhat$};
		\end{axis}
			\node[above right] at (0,2) {~The unit square $U$};
	\end{tikzpicture}
	~~~~
	\begin{tikzpicture}
		\begin{axis}[
		    anchor=origin,
		    disabledatascaling,
		    xmin=-2,xmax=3,
		    ymin=-1,ymax=2,
			xtick={-2,...,3},
			ytick={-2,...,3},
		    x=1cm,y=1cm,
		    grid=both,
		    grid style={line width=.1pt, draw=gray!10},
		    %major grid style={line width=.2pt,draw=gray!50},
		    axis lines=middle,
		    minor tick num=0,
		    enlargelimits={abs=0.5},
		    axis line style={latex-latex},
		    ticklabel style={font=\tiny,fill=white},
		    xlabel style={at={(ticklabel* cs:1)},anchor=north west},
		    ylabel style={at={(ticklabel* cs:1)},anchor=south west}
		]
			\fill [mygreen, opacity=.3] (1,2) -- (3,0) -- (1,-1) node[midway, below right, opacity=1] {$P$} -- (-1,1) -- cycle;
			\draw [black!70!white, thick,dashed, ->] (0,0) -- (2,1) node[midway, below right] {$\vec a$};
			\draw [black!70!white, thick,dashed, ->] (0,0) -- (-1,1) node[midway, left, yshift=-3pt] {$\vec b$};
		\end{axis}
			\node[above right] at (0,2) {~The parallelogram $P$};
	\end{tikzpicture}
\end{center}

Each set so far is a set of linear combinations, and we 
have made different shapes by restricting the coefficients of those linear
combinations. There are two ways of restricting linear combinations that arise
often enough to get their own names.

\SavedDefinitionRender{NonnegativeConvexLinearCombinations}

You can think of a non-negative linear combinations as vector you can arrive at by
only displacing ``forward''.
Convex linear combinations can be thought of as weighted averages of vectors (the average of $\vec v_1,\ldots,
\vec v_n$ would be the convex linear combination with coefficients $\alpha_i=\frac{1}{n}$). 
A convex linear combination
of two vectors gives a point on the line segment connecting them. 

\begin{example}
	Let $\vec a=\mat{2\\1}$ and $\vec b=\mat{-1\\1}$ and define
	\begin{align*}
		A&=\Set{\vec x\given \vec x\text{ is a convex linear combination of }\vec a\text{ and }\vec b}\\
		&=\Set{\vec x\given \vec x=\alpha\vec a+(1-\alpha)\vec b\text{ for some }\alpha\in [0,1]}.
	\end{align*}
	Draw $A$.

	We know $\vec x=\alpha\vec a+(1-\alpha)\vec b\in A$ whenever $\alpha\in[0,1]$. If we rearrange the
	equation $\vec x=\alpha\vec a+(1-\alpha)\vec b$, we see
	\[
		\vec x=\alpha\vec a-\alpha\vec b+\vec b = \alpha(\vec a-\vec b)+\vec b,
	\]
	which looks like the vector form of a line which passes through $\vec b$ with direction $\vec a-\vec b$.
	However, we have the additional restriction $\alpha\in[0,1]$, so $A$ is only the part of that line which connects $\vec a$
	and $\vec b$.

\begin{center}
	\usetikzlibrary{decorations.markings}
	\tikzset{test/.style n args={3}{
	    postaction={
	    decorate,
	    decoration={
	    markings,
	    mark=between positions 0 and \pgfdecoratedpathlength step 0.5pt with {
	    \pgfmathsetmacro\myval{multiply(
		divide(
		\pgfkeysvalueof{/pgf/decoration/mark info/distance from start}, \pgfdecoratedpathlength
		),
		100
	    )};
	    \pgfsetfillcolor{#3!\myval!#2};
	    \pgfpathcircle{\pgfpointorigin}{#1};
	    \pgfusepath{fill};}
	}}}}
	\begin{tikzpicture}
		\begin{axis}[
		    anchor=origin,
		    name=plot1,
		    disabledatascaling,
		    xmin=-1,xmax=3,
		    ymin=-1,ymax=2,
			xtick={-4,...,4},
			ytick={-2,...,4},
		    x=1cm,y=1cm,
		    grid=both,
		    grid style={line width=.1pt, draw=black!10},
		    %major grid style={line width=.2pt,draw=gray!50},
		    axis lines=middle,
		    minor tick num=0,
		    enlargelimits={abs=0.5},
		    axis line style={latex-latex},
		    ticklabel style={font=\tiny,fill=\currentbackgroundcolor},
		    xlabel style={at={(ticklabel* cs:1)},anchor=north west},
		    ylabel style={at={(ticklabel* cs:1)},anchor=south west}
		]
			\coordinate (A) at (2,1);
			\coordinate (B) at (1,-1);
			\coordinate (C) at ($.3*(A)+.7*(B)$);
			\coordinate (D) at ($.7*(A)+.3*(B)$);
			%\fill[mypink] (A) circle (2pt) node[above right] {$\vec a$};
			%\fill[mygreen] (B) circle (2pt) node[above right] {$\vec b$};
			\draw[thick, mypink, ->] (0,0) -- (A) ;
			\draw[thick, mypink!30!mygreen, ->] (0,0) -- (C) ;
			\draw[thick, mypink!70!mygreen, ->] (0,0) -- (D) ;
			\draw[thick, mygreen, ->] (0,0) -- (B);
		\end{axis}
		\node[right, mypink] at (A) {\small$\vec a=1\vec a+(1-1)\vec b$};
		\node[right, mypink!30!mygreen] at (C) {\small$\tfrac{1}{3}\vec a+(1-\tfrac{1}{3})\vec b$};
		\node[right, mypink!70!mygreen] at (D) {\small$\tfrac{2}{3}\vec a+(1-\tfrac{2}{3})\vec b$};
		\node[right, mygreen] at (B) {\small$\vec b=0\vec a+(1-0)\vec b$};
	\end{tikzpicture}
	\begin{tikzpicture}
		\begin{axis}[
		    anchor=origin,
		    name=plot1,
		    disabledatascaling,
		    xmin=-1,xmax=3,
		    ymin=-1,ymax=2,
			xtick={-4,...,4},
			ytick={-2,...,4},
		    x=1cm,y=1cm,
		    grid=both,
		    grid style={line width=.1pt, draw=black!10},
		    %major grid style={line width=.2pt,draw=gray!50},
		    axis lines=middle,
		    minor tick num=0,
		    enlargelimits={abs=0.5},
		    axis line style={latex-latex},
		    ticklabel style={font=\tiny,fill=\currentbackgroundcolor},
		    xlabel style={at={(ticklabel* cs:1)},anchor=north west},
		    ylabel style={at={(ticklabel* cs:1)},anchor=south west}
		]
			\coordinate (A) at (2,1);
			\coordinate (B) at (1,-1);
			\coordinate (C) at ($.3*(A)+.7*(B)$);
			\coordinate (D) at ($.7*(A)+.3*(B)$);
			\fill[mypink] (A) circle (2pt);
			\fill[mygreen] (B) circle (2pt);
			%\draw[thick, mypink, ->] (0,0) -- (A) ;
			%\draw[thick, mypink!30!mygreen, ->] (0,0) -- (C) ;
			%\draw[thick, mypink!70!mygreen, ->] (0,0) -- (D) ;
			%\draw[thick, mygreen, ->] (0,0) -- (B) node[below right] {$\vec b=0\vec a+(1-0)\vec b$};
		\end{axis}
		\draw [test={0.8pt}{mypink}{mygreen}, thick] (A) to node[midway, above left, mygreen!40!mypink] {$A$}(B) ;
		\node[above right, mypink] at (A) {$\vec a$};
		\node[below right, mygreen] at (B) {$\vec b$};
	\end{tikzpicture}
\end{center}
Since $A$ is an infinite collection of vectors, it's better to draw vectors in $A$ as dots rather than lines from the origin.
\end{example}

